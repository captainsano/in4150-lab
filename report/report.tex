\documentclass [a4,twoside,11pt] {article}

\usepackage{times}
\usepackage{color}
\usepackage{longtable}
\usepackage{graphicx,rotating,booktabs}

\setlength{\textwidth}{6in}
\setlength{\textheight}{9in}
\setlength{\oddsidemargin}{0.25in}
\setlength{\evensidemargin}{0.25in}

\hyphenation{practicum-ad-mini-stratie}

\begin{document}

\title{
    \flushleft
      {\small 
      {Distributed Systems Group \\
        Department Software Technology\\
      Faculty EEMCS\\
      DELFT UNIVERSITY OF TECHNOLOGY\\}}
    \center {
      \LARGE{LAB REPORT \\ DISTRIBUTED ALGORITHMS (IN4150) \\ 2017-2018 \\}
      \vspace{0.5cm}
      Randomized Byzantine Agreement
    }
}
\author{
  \newblock Santhos Baala Ramalingam Santhanakrishnan \\ 4740270 \\ \\
  \newblock Chuting Zhu \\ 4728572 \\
}
\date{December 21, 2017}
\maketitle

\setcounter{secnumdepth}{1}

% ============================================================================

The following sections tabulate results from the various cases that we tried. The table columns
correspond to process ids, red indicates malicious process and the rows correspond to the round-phase, 1N for 1st round notification phase, 1P for 1st round proposal phase and so on. The consensus reached is indicate in bold at the end of each tabulation. For malicious process, since there can be multiple broadcasts or no broadcasts in a single round, we list only the sampled values.

\section*{Case 1: 6 processors with 1 faulty}

\begin{center}
\small{Table 1. Case 1 - \textbf{Consensus value: 1}, \textbf{Total Rounds: 41}}
\vspace{-0.5cm}
\end{center}
\small
\begin{longtable}{ccccccc}
  \hline
  & \textcolor{red}{1} & 2 & 3 & 4 & 5 & 6 \\
  \hline
  1N & 1 & 0 & 1 & 0 & 1 & 0 \\
1P & 6 & 7 & 7 & 4 & 4 & 1 \\
\hline
2N & 0 & 0 & 1 & 0 & 0 & 0 \\
2P & 5 & 2 & 0 & 0 & 0 & 6 \\
\hline
3N & 1 & 0 & 1 & 0 & 0 & 1 \\
3P & 6 & 5 & 5 & 4 & 7 & 0 \\
\hline
4N &   & 0 & 0 & 1 & 1 & 0 \\
4P & 5 & 8 & 8 & 1 & 3 & 8 \\
\hline
5N & 1 & 1 & 1 & 0 & 0 & 0 \\
5P & 1 & 0 & 7 & 0 & 0 & 5 \\
\hline
6N & 1 & 0 & 0 & 1 & 1 & 0 \\
6P & 5 & 8 & 3 & 5 & 6 & 2 \\
\hline
7N & 1 & 0 & 1 & 0 & 0 & 0 \\
7P &   & 2 & 0 & 5 & 6 & 0 \\
\hline
8N & 0 & 1 & 1 & 1 & 1 & 0 \\
8P & 1 & 7 & 3 & 3 & 8 & 1 \\
\hline
9N & 1 & 1 & 0 & 1 & 1 & 0 \\
9P & 1 & 6 & 1 & 3 & 6 & 9 \\
\hline
10N & 0 & 1 & 1 & 1 & 1 & 1 \\
10P & 0 & 6 & 5 & 1 & 5 & 9 \\
\hline
11N & 1 & 0 & 1 & 0 & 0 & 1 \\
11P &   & 8 & 5 & 6 & 4 & 9 \\
\hline
12N & 1 & 1 & 0 & 0 & 0 & 1 \\
12P & 4 & 6 & 3 & 1 & 7 & 3 \\
\hline
13N & 1 & 1 & 0 & 1 & 1 & 0 \\
13P & 1 & 5 & 1 & 6 & 0 & 1 \\
\hline
14N &   & 1 & 0 & 1 & 1 & 1 \\
14P & 0 & 6 & 1 & 2 & 5 & 2 \\
\hline
15N & 0 & 0 & 0 & 0 & 0 & 1 \\
15P & 6 & 0 & 3 & 0 & 1 & 0 \\
\hline
16N & 1 & 0 & 0 & 0 & 0 & 0 \\
16P & 5 & 0 & 4 & 0 & 0 & 0 \\
\hline
17N & 1 & 0 & 0 & 0 & 0 & 1 \\
17P &   & 0 & 7 & 0 & 5 & 8 \\
\hline
18N & 1 & 1 & 0 & 1 & 0 & 1 \\
18P & 9 & 1 & 1 & 0 & 2 & 9 \\
\hline
19N &   & 1 & 0 & 1 & 1 & 1 \\
19P &   & 9 & 1 & 7 & 9 & 2 \\
\hline
20N & 0 & 1 & 0 & 1 & 1 & 1 \\
20P &   & 5 & 4 & 0 & 3 & 3 \\
\hline
21N & 1 & 1 & 1 & 0 & 0 & 0 \\
21P &   & 1 & 9 & 9 & 2 & 3 \\
\hline
22N & 1 & 1 & 0 & 0 & 0 & 0 \\
22P & 5 & 9 & 7 & 1 & 7 & 0 \\
\hline
23N & 1 & 0 & 0 & 0 & 0 & 1 \\
23P & 0 & 9 & 5 & 5 & 7 & 0 \\
\hline
24N & 1 & 0 & 0 & 1 & 1 & 0 \\
24P & 2 & 1 & 7 & 0 & 6 & 5 \\
\hline
25N & 1 & 1 & 0 & 1 & 0 & 0 \\
25P &   & 0 & 5 & 3 & 8 & 9 \\
\hline
26N & 0 & 0 & 1 & 1 & 0 & 1 \\
26P &   & 5 & 0 & 7 & 4 & 8 \\
\hline
27N & 0 & 0 & 0 & 0 & 0 & 1 \\
27P &   & 0 & 8 & 0 & 2 & 0 \\
\hline
28N & 1 & 0 & 0 & 0 & 0 & 1 \\
28P &   & 3 & 0 & 9 & 0 & 0 \\
\hline
29N & 1 & 0 & 1 & 0 & 1 & 0 \\
29P & 7 & 7 & 6 & 0 & 4 & 9 \\
\hline
30N & 0 & 1 & 1 & 1 & 1 & 0 \\
30P & 7 & 5 & 5 & 9 & 4 & 0 \\
\hline
31N & 0 & 1 & 1 & 1 & 1 & 0 \\
31P & 3 & 2 & 4 & 0 & 5 & 1 \\
\hline
32N &   & 1 & 0 & 1 & 1 & 0 \\
32P & 6 & 6 & 7 & 5 & 1 & 8 \\
\hline
33N & 0 & 1 & 1 & 1 & 1 & 0 \\
33P & 9 & 8 & 3 & 4 & 7 & 1 \\
\hline
34N &   & 1 & 0 & 1 & 0 & 1 \\
34P &   & 4 & 1 & 3 & 2 & 1 \\
\hline
35N & 0 & 1 & 1 & 0 & 1 & 0 \\
35P & 3 & 1 & 5 & 0 & 6 & 9 \\
\hline
36N &   & 0 & 0 & 1 & 0 & 0 \\
36P & 7 & 2 & 4 & 0 & 6 & 2 \\
\hline
37N &   & 0 & 1 & 0 & 0 & 0 \\
37P &   & 7 & 0 & 3 & 6 & 1 \\
\hline
38N &   & 1 & 0 & 0 & 1 & 0 \\
38P &   & 6 & 5 & 7 & 5 & 7 \\
\hline
39N & 0 & 0 & 1 & 1 & 1 & 1 \\
39P & 5 & 6 & 5 & 3 & 9 & 6 \\
\hline
40N & 0 & 1 & 0 & 1 & 1 & 1 \\
40P & 7 & 1 & 1 & 9 & 2 & 9 \\
\hline
41N &   & 1 & 1 & 1 & 1 & 1 \\
41P & 1 & 1 & 1 & 1 & 1 & 1 \\
\hline
\end{longtable}

\section*{Case 2: 11 processors with 2 faulty}

\begin{center}
\small{Table 2. Case 2 - \textbf{Consensus value: 1}, \textbf{Total Rounds: 12}}
\vspace{-0.5cm}
\end{center}
\small
\begin{longtable}{cccccccccccc}
  \hline
  & \textcolor{red}{1} & \textcolor{red}{2} & 3 & 4 & 5 & 6 & 7 & 8 & 9 & 10 & 11\\
  \hline
  1N & 0 & 0 & 1 & 0 & 1 & 0 & 1 & 0 & 1 & 1 & 0 \\
1P & 2 & 2 & 6 & 4 & 7 & 7 & 4 & 8 & 0 & 2 & 7 \\
\hline
2N &   & 1 & 0 & 1 & 0 & 1 & 0 & 1 & 1 & 1 & 0 \\
2P &   &   & 4 & 0 & 1 & 2 & 1 & 8 & 9 & 7 & 2 \\
\hline
3N & 0 &   & 0 & 0 & 1 & 1 & 1 & 1 & 0 & 0 & 1 \\
3P &   & 3 & 0 & 5 & 7 & 0 & 9 & 3 & 2 & 0 & 7 \\
\hline
4N &   &   & 0 & 0 & 1 & 0 & 0 & 0 & 0 & 0 & 1 \\
4P & 7 & 6 & 3 & 9 & 0 & 6 & 8 & 2 & 4 & 3 & 0 \\
\hline
5N &   & 0 & 1 & 0 & 1 & 0 & 1 & 0 & 1 & 1 & 0 \\
5P & 2 &   & 2 & 6 & 6 & 2 & 9 & 9 & 1 & 0 & 2 \\
\hline
6N & 0 & 1 & 1 & 1 & 1 & 1 & 1 & 0 & 1 & 1 & 0 \\
6P & 4 &   & 6 & 4 & 0 & 7 & 9 & 9 & 1 & 1 & 5 \\
\hline
7N & 0 & 0 & 0 & 1 & 1 & 0 & 1 & 1 & 1 & 0 & 0 \\
7P & 5 & 7 & 3 & 3 & 2 & 9 & 4 & 5 & 8 & 7 & 3 \\
\hline
8N &   &   & 1 & 1 & 1 & 0 & 1 & 1 & 1 & 1 & 0 \\
8P & 4 & 0 & 1 & 8 & 0 & 1 & 1 & 6 & 2 & 8 & 1 \\
\hline
9N & 0 & 0 & 0 & 0 & 0 & 1 & 0 & 0 & 0 & 1 & 1 \\
9P &   &   & 1 & 9 & 9 & 6 & 2 & 5 & 2 & 9 & 1 \\
\hline
10N & 0 & 0 & 0 & 0 & 1 & 1 & 1 & 1 & 1 & 0 & 0 \\
10P & 8 & 4 & 6 & 0 & 5 & 2 & 5 & 2 & 9 & 3 & 1 \\
\hline
11N & 0 &   & 1 & 1 & 1 & 1 & 0 & 0 & 0 & 0 & 1 \\
11P & 1 & 1 & 6 & 2 & 9 & 8 & 8 & 6 & 7 & 4 & 0 \\
\hline
12N &   &   & 1 & 1 & 1 & 1 & 1 & 1 & 1 & 1 & 0 \\
12P & 6 &   & 1 & 1 & 1 & 1 & 1 & 1 & 1 & 1 & 1 \\
\hline

\end{longtable}

\newpage

\section*{Case 3: 40 processors with 7 faulty}

\begin{center}
\small{Table 3. Case 3 - \textbf{Consensus value: 0}, \textbf{Total Rounds: 298}}
\vspace{-0.5cm}
\end{center}
\small\setlength\tabcolsep{1pt}
\begin{longtable}{cccccccccccccccccccccccccccccccccccccccccccccccccccccccc|}
  \hline
  & \textcolor{red}{1} & \textcolor{red}{2} & \textcolor{red}{3} & \textcolor{red}{4} & \textcolor{red}{5} & \textcolor{red}{6} & \textcolor{red}{7} & 8 & 9 & 10 & 11 & 12 & 13 & 14 & 15 & 16 & 17 & 18 & 19 & 20 & 21 & 22 & 23 & 24 & 25 & 26 & 27 & 28 & 29 & 30 & 31 & 32 & 33 & 34 & 35 & 36 & 37 & 38 & 39 & 40 \\
  \hline
  1N & 0 & 1 & 0 & 1 & 0 & 1 & 1 & 1 & 0 & 1 & 0 & 0 & 0 & 1 & 0 & 1 & 0 & 0 & 0 & 0 & 1 & 1 & 0 & 0 & 1 & 0 & 1 & 1 & 0 & 0 & 1 & 0 & 1 & 0 & 0 & 1 & 0 & 1 & 0 & 0 \\
1P & 5 & 5 & 2 & 7 & 8 & 0 & 0 & 0 & 5 & 0 & 0 & 0 & 2 & 3 & 7 & 5 & 0 & 2 & 1 & 7 & 1 & 4 & 1 & 2 & 1 & 4 & 5 & 6 & 1 & 4 & 2 & 6 & 1 & 3 & 6 & 7 & 7 & 0 & 7 & 8 \\
\hline
2N & 1 & 0 & 0 & 1 & 1 & 1 & 1 & 1 & 0 & 0 & 1 & 1 & 1 & 1 & 0 & 1 & 0 & 1 & 1 & 0 & 0 & 0 & 1 & 1 & 1 & 1 & 0 & 0 & 0 & 1 & 1 & 0 & 0 & 0 & 0 & 0 & 1 & 0 & 1 & 0 \\
2P & 7 &   & 9 & 4 & 4 & 2 & 3 & 0 & 6 & 5 & 8 & 4 & 3 & 5 & 4 & 7 & 0 & 9 & 0 & 7 & 2 & 6 & 5 & 1 & 3 & 1 & 7 & 8 & 3 & 6 & 3 & 7 & 8 & 4 & 5 & 6 & 1 & 2 & 2 & 0 \\
\hline
3N & 0 & 1 & 0 & 1 & 1 & 0 & 1 & 1 & 1 & 0 & 0 & 1 & 1 & 0 & 0 & 0 & 1 & 0 & 0 & 1 & 0 & 0 & 1 & 1 & 0 & 1 & 1 & 0 & 1 & 1 & 0 & 1 & 0 & 1 & 0 & 0 & 1 & 1 & 0 & 1 \\
3P & 0 & 7 & 1 & 4 & 8 & 6 & 3 & 4 & 5 & 3 & 1 & 0 & 1 & 5 & 0 & 2 & 3 & 5 & 2 & 7 & 0 & 5 & 8 & 1 & 3 & 7 & 1 & 5 & 1 & 2 & 8 & 5 & 8 & 9 & 9 & 7 & 4 & 1 & 5 & 4 \\
\hline
4N & 0 & 0 & 1 & 1 & 0 & 1 & 0 & 0 & 1 & 0 & 0 & 0 & 1 & 0 & 0 & 1 & 1 & 0 & 0 & 0 & 0 & 1 & 1 & 0 & 1 & 1 & 1 & 1 & 1 & 0 & 0 & 0 & 1 & 0 & 0 & 0 & 0 & 1 & 0 & 0 \\
4P & 5 & 1 & 5 & 4 & 5 & 4 & 3 & 2 & 8 & 2 & 6 & 0 & 6 & 0 & 9 & 3 & 1 & 0 & 3 & 4 & 3 & 6 & 9 & 4 & 2 & 3 & 7 & 6 & 7 & 0 & 1 & 8 & 1 & 4 & 3 & 3 & 5 & 3 & 4 & 0 \\
\hline
5N & 1 & 1 & 1 & 0 & 1 & 1 & 1 & 1 & 1 & 0 & 1 & 1 & 1 & 0 & 1 & 0 & 0 & 1 & 1 & 0 & 1 & 0 & 1 & 0 & 0 & 1 & 0 & 0 & 1 & 1 & 0 & 0 & 0 & 1 & 0 & 0 & 1 & 1 & 1 & 0 \\
5P & 9 & 5 & 8 & 1 & 4 & 7 & 0 & 3 & 1 & 6 & 9 & 6 & 9 & 4 & 8 & 9 & 8 & 4 & 1 & 9 & 0 & 5 & 5 & 4 & 8 & 8 & 0 & 0 & 0 & 5 & 3 & 3 & 0 & 6 & 8 & 4 & 3 & 7 & 0 & 0 \\
\hline
6N & 0 & 0 & 0 & 1 & 1 & 1 & 0 & 0 & 1 & 1 & 0 & 1 & 0 & 0 & 0 & 1 & 1 & 0 & 0 & 1 & 1 & 0 & 1 & 0 & 0 & 0 & 1 & 1 & 0 & 1 & 1 & 1 & 1 & 1 & 1 & 1 & 0 & 1 & 0 & 1 \\
6P & 6 & 6 & 3 & 7 & 9 & 9 & 3 & 0 & 5 & 3 & 6 & 1 & 8 & 7 & 7 & 3 & 9 & 6 & 3 & 9 & 6 & 1 & 4 & 8 & 7 & 9 & 9 & 3 & 5 & 1 & 9 & 7 & 6 & 7 & 1 & 1 & 8 & 4 & 4 & 3 \\
\hline
7N & 0 & 0 & 1 & 1 & 1 & 0 & 0 & 0 & 0 & 1 & 0 & 0 & 0 & 1 & 1 & 0 & 0 & 1 & 0 & 1 & 0 & 0 & 0 & 0 & 0 & 0 & 1 & 0 & 0 & 0 & 0 & 0 & 1 & 1 & 1 & 0 & 0 & 1 & 1 & 0 \\
7P & 8 & 5 & 9 & 9 & 7 & 3 & 3 & 5 & 3 & 4 & 1 & 6 & 6 & 5 & 3 & 4 & 4 & 4 & 2 & 5 & 7 & 0 & 3 & 3 & 9 & 6 & 9 & 9 & 4 & 7 & 1 & 8 & 4 & 6 & 2 & 6 & 3 & 7 & 9 & 4 \\
\hline
8N & 1 & 1 & 1 & 0 & 0 & 0 & 1 & 0 & 1 & 0 & 1 & 1 & 1 & 0 & 1 & 1 & 0 & 0 & 1 & 1 & 0 & 1 & 1 & 1 & 1 & 0 & 0 & 0 & 0 & 1 & 0 & 1 & 1 & 0 & 0 & 1 & 0 & 1 & 0 & 0 \\
8P & 9 & 6 & 6 & 3 & 9 & 5 & 3 & 9 & 6 & 0 & 8 & 0 & 8 & 5 & 4 & 0 & 3 & 7 & 7 & 4 & 2 & 8 & 4 & 3 & 7 & 0 & 7 & 5 & 6 & 2 & 8 & 2 & 8 & 3 & 3 & 3 & 8 & 4 & 6 & 9 \\
\hline
9N & 0 & 1 & 0 & 1 & 0 & 1 & 1 & 1 & 0 & 1 & 0 & 0 & 1 & 0 & 1 & 0 & 1 & 1 & 1 & 1 & 1 & 1 & 0 & 0 & 0 & 1 & 1 & 0 & 0 & 1 & 0 & 0 & 1 & 1 & 0 & 1 & 0 & 0 & 0 & 1 \\
9P & 1 & 4 & 4 & 8 & 9 & 1 & 0 & 2 & 0 & 3 & 7 & 4 & 3 & 7 & 1 & 8 & 0 & 4 & 7 & 8 & 1 & 1 & 9 & 8 & 0 & 4 & 5 & 8 & 3 & 5 & 0 & 3 & 8 & 1 & 6 & 6 & 2 & 5 & 6 & 9 \\
\hline
10N & 0 & 0 & 1 & 0 & 0 & 0 & 0 & 1 & 1 & 0 & 1 & 1 & 0 & 1 & 0 & 1 & 1 & 1 & 0 & 0 & 0 & 1 & 0 & 1 & 0 & 0 & 0 & 0 & 0 & 0 & 0 & 1 & 1 & 0 & 1 & 1 & 0 & 0 & 1 & 0 \\
10P & 8 & 7 & 9 & 4 & 0 & 4 & 9 & 0 & 4 & 9 & 1 & 4 & 9 & 9 & 8 & 6 & 5 & 2 & 5 & 4 & 9 & 4 & 0 & 0 & 4 & 4 & 2 & 7 & 9 & 0 & 5 & 9 & 7 & 6 & 4 & 7 & 4 & 6 & 8 & 3 \\
\hline
11N & 1 & 1 & 1 & 1 & 0 & 0 & 1 & 0 & 1 & 0 & 0 & 0 & 0 & 1 & 1 & 0 & 0 & 1 & 1 & 0 & 0 & 0 & 1 & 0 & 1 & 0 & 0 & 1 & 0 & 1 & 0 & 1 & 1 & 1 & 0 & 1 & 0 & 1 & 1 & 1 \\
11P & 2 & 6 & 6 & 2 & 7 & 2 & 1 & 7 & 8 & 7 & 9 & 8 & 1 & 0 & 4 & 2 & 3 & 0 & 6 & 3 & 5 & 0 & 3 & 1 & 0 & 9 & 9 & 5 & 7 & 0 & 6 & 5 & 0 & 7 & 8 & 7 & 8 & 3 & 4 & 3 \\
\hline
12N & 1 & 0 & 0 & 0 & 1 & 1 & 1 & 0 & 0 & 0 & 1 & 0 & 0 & 0 & 1 & 0 & 0 & 0 & 1 & 1 & 0 & 0 & 0 & 0 & 0 & 0 & 0 & 1 & 1 & 1 & 1 & 0 & 0 & 0 & 1 & 0 & 0 & 0 & 1 & 0 \\
12P & 9 & 0 & 8 & 7 & 6 & 1 & 3 & 4 & 0 & 9 & 2 & 7 & 2 & 2 & 0 & 9 & 9 & 2 & 4 & 8 & 3 & 9 & 4 & 3 & 9 & 1 & 4 & 7 & 8 & 4 & 4 & 8 & 8 & 0 & 6 & 0 & 6 & 5 & 3 & 4 \\
\hline
13N & 1 & 0 & 0 & 0 & 1 & 0 & 0 & 1 & 1 & 0 & 0 & 0 & 1 & 1 & 0 & 0 & 0 & 0 & 1 & 0 & 0 & 0 & 0 & 0 & 1 & 1 & 1 & 0 & 1 & 0 & 0 & 1 & 1 & 1 & 0 & 1 & 1 & 0 & 1 & 1 \\
13P & 6 & 6 & 4 & 9 & 9 & 4 & 9 & 6 & 7 & 0 & 1 & 7 & 6 & 4 & 4 & 6 & 6 & 8 & 6 & 0 & 0 & 2 & 1 & 1 & 9 & 2 & 1 & 1 & 8 & 1 & 4 & 1 & 9 & 8 & 7 & 1 & 1 & 9 & 0 & 6 \\
\hline
14N & 0 & 0 & 0 & 1 & 0 & 1 & 0 & 1 & 0 & 1 & 0 & 1 & 1 & 0 & 1 & 1 & 1 & 0 & 1 & 0 & 0 & 0 & 1 & 1 & 0 & 1 & 0 & 0 & 1 & 1 & 1 & 1 & 1 & 0 & 0 & 0 & 0 & 1 & 0 & 1 \\
14P & 3 & 4 & 0 & 4 & 8 & 9 & 2 & 5 & 8 & 0 & 4 & 2 & 0 & 8 & 0 & 0 & 3 & 3 & 8 & 9 & 5 & 0 & 0 & 6 & 6 & 4 & 0 & 1 & 7 & 3 & 6 & 8 & 1 & 1 & 7 & 6 & 6 & 2 & 1 & 3 \\
\hline
15N & 1 & 0 & 1 & 1 & 1 & 0 & 1 & 1 & 1 & 1 & 1 & 1 & 0 & 1 & 1 & 0 & 1 & 0 & 1 & 0 & 1 & 0 & 1 & 1 & 0 & 1 & 1 & 1 & 1 & 1 & 1 & 1 & 0 & 0 & 0 & 1 & 1 & 1 & 1 & 0 \\
15P & 4 & 6 & 5 & 2 & 8 & 7 & 8 & 5 & 1 & 9 & 6 & 3 & 8 & 2 & 0 & 1 & 4 & 7 & 0 & 2 & 6 & 8 & 9 & 5 & 7 & 8 & 1 & 9 & 3 & 1 & 4 & 2 & 4 & 7 & 2 & 3 & 1 & 7 & 8 & 1 \\
\hline
16N & 1 & 0 & 0 & 0 & 1 & 1 & 0 & 0 & 0 & 1 & 0 & 1 & 0 & 0 & 1 & 1 & 0 & 1 & 1 & 0 & 0 & 0 & 0 & 0 & 1 & 0 & 1 & 0 & 0 & 1 & 0 & 1 & 1 & 1 & 0 & 1 & 1 & 0 & 1 & 0 \\
16P & 9 & 6 & 0 & 0 & 6 & 1 & 3 & 3 & 4 & 2 & 7 & 6 & 1 & 8 & 5 & 7 & 8 & 6 & 4 & 7 & 5 & 1 & 7 & 4 & 9 & 6 & 6 & 5 & 0 & 7 & 4 & 3 & 4 & 1 & 7 & 1 & 1 & 8 & 7 & 4 \\
\hline
17N & 0 & 0 & 0 & 0 & 1 & 1 & 0 & 0 & 1 & 0 & 1 & 0 & 0 & 0 & 0 & 0 & 0 & 0 & 1 & 0 & 1 & 0 & 0 & 0 & 1 & 0 & 1 & 0 & 1 & 0 & 1 & 1 & 0 & 0 & 0 & 0 & 1 & 1 & 1 & 1 \\
17P & 0 & 5 & 1 & 2 & 4 & 6 & 2 & 0 & 0 & 3 & 6 & 9 & 1 & 2 & 7 & 2 & 4 & 7 & 0 & 0 & 0 & 1 & 6 & 7 & 3 & 4 & 0 & 5 & 7 & 8 & 7 & 6 & 9 & 2 & 7 & 6 & 1 & 6 & 1 & 8 \\
\hline
18N & 1 & 0 & 1 & 0 & 0 & 1 & 0 & 1 & 0 & 1 & 1 & 0 & 1 & 0 & 1 & 1 & 1 & 1 & 0 & 0 & 1 & 0 & 1 & 0 & 0 & 1 & 0 & 0 & 0 & 1 & 0 & 0 & 1 & 1 & 1 & 1 & 0 & 0 & 0 & 1 \\
18P & 4 & 4 & 0 & 0 & 9 & 8 & 7 & 3 & 6 & 5 & 4 & 9 & 1 & 6 & 4 & 8 & 1 & 7 & 2 & 9 & 6 & 1 & 3 & 7 & 9 & 1 & 1 & 0 & 3 & 7 & 8 & 5 & 1 & 5 & 4 & 6 & 5 & 4 & 9 & 0 \\
\hline
19N & 0 & 0 & 0 & 1 & 0 & 0 & 0 & 1 & 0 & 0 & 0 & 0 & 1 & 0 & 0 & 0 & 1 & 0 & 0 & 1 & 1 & 1 & 1 & 0 & 0 & 0 & 0 & 0 & 1 & 1 & 1 & 0 & 1 & 1 & 1 & 1 & 1 & 0 & 1 & 0 \\
19P & 6 & 8 & 2 & 9 & 3 & 2 & 5 & 2 & 0 & 2 & 7 & 8 & 2 & 1 & 2 & 4 & 6 & 3 & 4 & 0 & 1 & 3 & 5 & 6 & 5 & 9 & 1 & 2 & 8 & 5 & 5 & 9 & 5 & 1 & 7 & 0 & 8 & 9 & 3 & 8 \\
\hline
20N & 0 & 0 & 0 & 1 & 1 & 0 & 1 & 1 & 1 & 1 & 0 & 0 & 0 & 0 & 0 & 0 & 0 & 1 & 0 & 1 & 0 & 0 & 0 & 0 & 1 & 0 & 1 & 1 & 1 & 1 & 1 & 0 & 0 & 1 & 1 & 0 & 0 & 0 & 0 & 1 \\
20P & 2 & 8 & 7 & 5 & 6 & 1 & 7 & 5 & 4 & 8 & 5 & 8 & 0 & 7 & 9 & 4 & 8 & 2 & 5 & 8 & 5 & 0 & 3 & 5 & 7 & 8 & 7 & 8 & 7 & 9 & 7 & 8 & 2 & 7 & 5 & 6 & 2 & 6 & 7 & 8 \\
\hline
21N & 0 & 1 & 1 & 1 & 0 & 0 & 1 & 0 & 0 & 1 & 0 & 0 & 0 & 0 & 1 & 1 & 0 & 1 & 1 & 0 & 0 & 0 & 0 & 1 & 0 & 0 & 0 & 1 & 1 & 0 & 0 & 1 & 1 & 0 & 1 & 1 & 1 & 0 & 1 & 0 \\
21P & 1 & 4 & 8 & 0 & 2 & 8 & 9 & 8 & 6 & 8 & 0 & 2 & 8 & 8 & 6 & 2 & 3 & 4 & 0 & 5 & 5 & 3 & 5 & 4 & 0 & 6 & 7 & 5 & 0 & 5 & 4 & 0 & 6 & 8 & 7 & 3 & 1 & 7 & 5 & 2 \\
\hline
22N & 1 & 1 & 0 & 0 & 0 & 1 & 1 & 1 & 0 & 1 & 1 & 1 & 1 & 1 & 0 & 0 & 1 & 1 & 1 & 1 & 0 & 0 & 0 & 0 & 0 & 0 & 0 & 1 & 0 & 1 & 0 & 0 & 1 & 1 & 1 & 1 & 1 & 1 & 1 & 1 \\
22P & 6 & 7 & 1 & 7 & 8 & 4 & 2 & 5 & 8 & 1 & 4 & 1 & 2 & 5 & 0 & 8 & 8 & 0 & 1 & 0 & 3 & 7 & 5 & 3 & 7 & 5 & 9 & 1 & 5 & 4 & 3 & 8 & 9 & 7 & 2 & 1 & 1 & 2 & 7 & 5 \\
\hline
23N & 1 & 1 & 0 & 0 & 1 & 1 & 0 & 0 & 1 & 0 & 0 & 1 & 1 & 1 & 1 & 0 & 1 & 0 & 1 & 0 & 0 & 1 & 0 & 1 & 0 & 1 & 1 & 0 & 1 & 0 & 0 & 1 & 1 & 1 & 1 & 1 & 0 & 0 & 1 & 0 \\
23P & 0 & 2 & 1 & 7 & 9 & 9 & 4 & 7 & 5 & 2 & 3 & 9 & 8 & 3 & 5 & 0 & 4 & 8 & 3 & 2 & 5 & 8 & 1 & 4 & 7 & 0 & 9 & 4 & 5 & 8 & 0 & 9 & 5 & 2 & 1 & 3 & 3 & 3 & 9 & 7 \\
\hline
24N & 0 & 1 & 1 & 0 & 1 & 0 & 0 & 0 & 0 & 1 & 1 & 0 & 1 & 0 & 0 & 1 & 0 & 0 & 0 & 1 & 0 & 0 & 0 & 0 & 1 & 1 & 0 & 0 & 1 & 0 & 1 & 1 & 0 & 0 & 0 & 0 & 1 & 0 & 1 & 1 \\
24P & 8 & 7 & 4 & 2 & 6 & 3 & 7 & 8 & 5 & 4 & 3 & 9 & 3 & 9 & 3 & 8 & 2 & 4 & 7 & 6 & 7 & 0 & 0 & 8 & 6 & 4 & 1 & 9 & 1 & 9 & 5 & 2 & 3 & 6 & 9 & 2 & 3 & 2 & 7 & 5 \\
\hline
25N & 1 & 1 & 1 & 0 & 1 & 0 & 0 & 1 & 1 & 1 & 1 & 0 & 1 & 0 & 1 & 1 & 1 & 0 & 1 & 0 & 1 & 1 & 0 & 1 & 0 & 1 & 1 & 0 & 1 & 1 & 0 & 1 & 1 & 0 & 0 & 0 & 1 & 0 & 0 & 1 \\
25P & 1 & 9 & 9 & 8 & 5 & 1 & 5 & 9 & 5 & 4 & 1 & 5 & 5 & 2 & 8 & 2 & 1 & 4 & 7 & 4 & 8 & 0 & 4 & 7 & 2 & 1 & 5 & 8 & 7 & 7 & 2 & 8 & 3 & 4 & 7 & 5 & 0 & 9 & 4 & 5 \\
\hline
26N & 0 & 0 & 1 & 1 & 0 & 1 & 1 & 0 & 1 & 0 & 1 & 0 & 1 & 0 & 0 & 0 & 0 & 1 & 0 & 0 & 0 & 0 & 0 & 1 & 0 & 0 & 1 & 1 & 0 & 0 & 1 & 0 & 1 & 0 & 1 & 1 & 0 & 1 & 1 & 1 \\
26P & 0 & 0 & 6 & 7 & 8 & 2 & 2 & 2 & 4 & 9 & 7 & 5 & 3 & 4 & 9 & 5 & 6 & 4 & 1 & 6 & 9 & 2 & 5 & 5 & 1 & 8 & 6 & 4 & 5 & 7 & 3 & 9 & 5 & 8 & 9 & 8 & 2 & 2 & 2 & 3 \\
\hline
27N & 1 & 0 & 0 & 0 & 1 & 1 & 0 & 0 & 1 & 0 & 1 & 1 & 1 & 1 & 1 & 1 & 0 & 1 & 0 & 1 & 0 & 1 & 1 & 1 & 1 & 0 & 0 & 1 & 0 & 0 & 0 & 1 & 0 & 0 & 1 & 1 & 0 & 1 & 0 & 1 \\
27P & 8 & 6 & 5 & 1 & 7 & 6 & 5 & 2 & 3 & 6 & 4 & 1 & 3 & 8 & 1 & 0 & 6 & 4 & 9 & 5 & 1 & 6 & 5 & 3 & 3 & 0 & 0 & 6 & 6 & 0 & 1 & 3 & 5 & 2 & 4 & 8 & 0 & 2 & 4 & 5 \\
\hline
28N & 1 & 0 & 1 & 1 & 1 & 0 & 1 & 0 & 1 & 1 & 1 & 1 & 0 & 1 & 0 & 1 & 0 & 0 & 1 & 1 & 0 & 0 & 1 & 1 & 1 & 1 & 0 & 0 & 0 & 1 & 0 & 0 & 1 & 1 & 0 & 1 & 1 & 0 & 1 & 0 \\
28P & 6 & 6 & 8 & 8 & 9 & 1 & 8 & 0 & 5 & 1 & 9 & 6 & 2 & 6 & 7 & 1 & 8 & 4 & 5 & 0 & 9 & 7 & 1 & 6 & 4 & 4 & 7 & 7 & 0 & 8 & 9 & 3 & 6 & 5 & 6 & 8 & 1 & 7 & 8 & 8 \\
\hline
29N & 0 & 0 & 0 & 0 & 0 & 1 & 0 & 0 & 0 & 1 & 0 & 1 & 1 & 1 & 0 & 1 & 1 & 1 & 1 & 1 & 1 & 1 & 0 & 0 & 1 & 1 & 0 & 1 & 0 & 1 & 1 & 0 & 0 & 1 & 1 & 0 & 0 & 1 & 0 & 1 \\
29P & 1 & 4 & 4 & 7 & 4 & 8 & 8 & 1 & 6 & 1 & 5 & 8 & 9 & 9 & 0 & 0 & 1 & 2 & 5 & 1 & 5 & 4 & 2 & 6 & 6 & 9 & 2 & 0 & 4 & 4 & 1 & 9 & 6 & 4 & 0 & 2 & 5 & 3 & 0 & 9 \\
\hline
30N & 1 & 0 & 1 & 0 & 1 & 0 & 1 & 0 & 0 & 0 & 0 & 0 & 0 & 0 & 0 & 1 & 1 & 0 & 0 & 1 & 0 & 0 & 1 & 1 & 1 & 1 & 0 & 0 & 0 & 0 & 1 & 1 & 1 & 0 & 1 & 0 & 1 & 0 & 0 & 1 \\
30P & 0 & 5 & 1 & 9 & 1 & 8 & 6 & 2 & 5 & 9 & 4 & 3 & 9 & 9 & 3 & 2 & 3 & 3 & 4 & 8 & 6 & 3 & 7 & 8 & 3 & 0 & 0 & 0 & 8 & 9 & 9 & 9 & 3 & 7 & 7 & 2 & 4 & 1 & 2 & 0 \\
\hline
31N & 1 & 1 & 0 & 0 & 1 & 1 & 1 & 0 & 0 & 0 & 0 & 1 & 1 & 0 & 1 & 0 & 0 & 0 & 1 & 1 & 1 & 0 & 0 & 0 & 1 & 1 & 0 & 1 & 0 & 1 & 1 & 0 & 0 & 1 & 1 & 1 & 0 & 1 & 1 & 0 \\
31P & 7 & 7 & 2 & 2 & 5 & 7 & 0 & 0 & 5 & 5 & 4 & 7 & 6 & 8 & 9 & 6 & 3 & 6 & 3 & 4 & 1 & 0 & 7 & 9 & 0 & 0 & 5 & 5 & 9 & 3 & 9 & 7 & 5 & 4 & 7 & 6 & 9 & 2 & 4 & 1 \\
\hline
32N & 1 & 0 & 0 & 1 & 1 & 0 & 0 & 0 & 1 & 1 & 1 & 0 & 1 & 0 & 1 & 0 & 0 & 0 & 0 & 1 & 0 & 1 & 0 & 0 & 0 & 0 & 1 & 0 & 0 & 1 & 0 & 1 & 0 & 0 & 1 & 1 & 0 & 1 & 0 & 1 \\
32P & 9 & 3 & 6 & 8 & 4 & 7 & 7 & 5 & 3 & 4 & 7 & 0 & 5 & 4 & 8 & 1 & 0 & 9 & 0 & 4 & 8 & 1 & 4 & 4 & 5 & 1 & 1 & 0 & 5 & 6 & 0 & 5 & 6 & 4 & 8 & 1 & 0 & 2 & 2 & 1 \\
\hline
33N & 1 & 0 & 0 & 1 & 1 & 1 & 1 & 0 & 1 & 1 & 1 & 1 & 0 & 0 & 0 & 0 & 1 & 0 & 0 & 1 & 1 & 1 & 0 & 1 & 0 & 0 & 1 & 1 & 0 & 1 & 0 & 0 & 0 & 1 & 0 & 1 & 1 & 1 & 1 & 1 \\
33P & 2 &   & 0 & 8 & 1 & 1 & 7 & 9 & 3 & 3 & 9 & 7 & 5 & 7 & 1 & 4 & 9 & 0 & 5 & 6 & 1 & 3 & 7 & 6 & 4 & 6 & 2 & 9 & 3 & 9 & 9 & 6 & 7 & 0 & 7 & 3 & 6 & 4 & 8 & 3 \\
\hline
34N & 1 & 0 &   & 1 & 1 & 0 & 1 & 1 & 1 & 1 & 0 & 1 & 1 & 1 & 1 & 1 & 1 & 0 & 0 & 0 & 0 & 0 & 0 & 1 & 0 & 0 & 1 & 1 & 0 & 1 & 1 & 0 & 1 & 0 & 0 & 0 & 1 & 0 & 1 & 0 \\
34P & 6 & 5 & 9 & 9 & 4 & 0 & 6 & 9 & 7 & 6 & 7 & 3 & 1 & 7 & 4 & 5 & 4 & 6 & 9 & 4 & 2 & 8 & 3 & 3 & 2 & 1 & 6 & 0 & 1 & 8 & 5 & 9 & 1 & 9 & 8 & 7 & 9 & 3 & 0 & 2 \\
\hline
35N & 0 & 0 & 1 & 1 & 0 & 1 & 0 & 0 & 1 & 0 & 1 & 1 & 0 & 1 & 1 & 0 & 0 & 0 & 1 & 1 & 1 & 0 & 1 & 1 & 0 & 0 & 1 & 1 & 1 & 1 & 1 & 0 & 0 & 1 & 1 & 1 & 0 & 1 & 0 & 0 \\
35P & 4 & 5 & 4 & 8 & 7 & 0 & 5 & 5 & 0 & 9 & 5 & 4 & 1 & 9 & 5 & 1 & 4 & 7 & 5 & 8 & 2 & 9 & 0 & 3 & 8 & 8 & 3 & 2 & 3 & 1 & 7 & 8 & 8 & 5 & 8 & 6 & 9 & 8 & 8 & 4 \\
\hline
36N & 1 & 1 & 1 & 0 & 1 & 1 & 0 & 0 & 1 & 1 & 0 & 1 & 0 & 0 & 1 & 0 & 1 & 1 & 0 & 1 & 1 & 0 & 0 & 0 & 0 & 1 & 1 & 1 & 0 & 1 & 0 & 0 & 1 & 1 & 1 & 0 & 0 & 1 & 1 & 0 \\
36P & 6 & 6 & 3 & 5 & 4 & 8 & 4 & 7 & 2 & 1 & 7 & 8 & 1 & 8 & 1 & 5 & 5 & 2 & 5 & 3 & 6 & 5 & 8 & 9 & 0 & 4 & 5 & 5 & 4 & 1 & 7 & 6 & 4 & 0 & 4 & 6 & 8 & 9 & 5 & 1 \\
\hline
37N & 0 & 0 & 0 & 1 & 1 & 0 & 1 & 0 & 1 & 0 & 1 & 1 & 1 & 0 & 0 & 0 & 1 & 1 & 0 & 0 & 0 & 1 & 0 & 0 & 1 & 1 & 1 & 1 & 1 & 1 & 0 & 0 & 1 & 1 & 0 & 1 & 0 & 0 & 1 & 1 \\
37P & 2 & 3 & 8 & 5 & 1 & 9 & 2 & 3 & 2 & 8 & 8 & 0 & 1 & 2 & 4 & 4 & 9 & 0 & 1 & 9 & 1 & 7 & 2 & 9 & 7 & 3 & 0 & 5 & 9 & 3 & 0 & 1 & 5 & 3 & 4 & 1 & 8 & 8 & 6 & 7 \\
\hline
38N & 1 & 1 & 0 & 0 & 0 & 1 & 0 & 0 & 0 & 0 & 1 & 1 & 0 & 1 & 1 & 0 & 1 & 0 & 0 & 1 & 0 & 1 & 1 & 0 & 0 & 1 & 0 & 0 & 0 & 1 & 1 & 1 & 1 & 1 & 0 & 1 & 1 & 0 & 1 & 0 \\
38P & 1 & 0 & 6 & 6 & 3 & 4 & 2 & 3 & 0 & 7 & 8 & 5 & 4 & 1 & 0 & 8 & 5 & 4 & 1 & 5 & 4 & 7 & 0 & 6 & 4 & 1 & 5 & 6 & 5 & 3 & 9 & 9 & 1 & 5 & 2 & 1 & 6 & 1 & 0 & 7 \\
\hline
39N & 0 & 0 & 0 & 1 & 1 & 0 & 0 & 1 & 1 & 0 & 1 & 1 & 1 & 0 & 0 & 0 & 0 & 0 & 1 & 0 & 1 & 1 & 1 & 1 & 0 & 1 & 1 & 1 & 1 & 0 & 1 & 0 & 1 & 0 & 1 & 1 & 1 & 0 & 0 & 0 \\
39P & 1 & 0 & 5 & 5 & 6 & 7 & 1 & 2 & 2 & 5 & 9 & 6 & 9 & 2 & 9 & 6 & 0 & 6 & 7 & 2 & 5 & 8 & 3 & 6 & 1 & 8 & 2 & 4 & 8 & 1 & 3 & 7 & 0 & 7 & 8 & 3 & 3 & 6 & 5 & 3 \\
\hline
40N & 0 & 1 & 1 & 1 & 0 & 0 & 0 & 0 & 1 & 1 & 1 & 1 & 0 & 1 & 0 & 1 & 0 & 1 & 0 & 1 & 0 & 1 & 1 & 1 & 1 & 1 & 1 & 0 & 0 & 1 & 1 & 0 & 0 & 0 & 0 & 1 & 1 & 0 & 0 & 1 \\
40P & 0 & 4 & 1 & 2 & 2 & 2 & 7 & 4 & 8 & 0 & 6 & 1 & 6 & 8 & 1 & 6 & 6 & 2 & 7 & 9 & 9 & 1 & 9 & 2 & 8 & 0 & 4 & 1 & 8 & 9 & 2 & 0 & 3 & 8 & 8 & 7 & 8 & 1 & 0 & 6 \\
\hline
41N & 1 & 1 & 1 & 1 & 0 & 1 & 0 & 0 & 1 & 0 & 1 & 1 & 1 & 0 & 0 & 1 & 1 & 0 & 0 & 0 & 0 & 0 & 0 & 1 & 1 & 0 & 1 & 0 & 1 & 0 & 0 & 0 & 0 & 0 & 1 & 1 & 0 & 1 & 0 & 1 \\
41P & 2 & 4 & 5 & 6 & 3 & 7 & 3 & 6 & 7 & 8 & 9 & 7 & 8 & 1 & 9 & 0 & 4 & 5 & 6 & 7 & 2 & 6 & 6 & 7 & 7 & 9 & 4 & 8 & 7 & 6 & 9 & 4 & 2 & 1 & 0 & 4 & 0 & 0 & 1 & 2 \\
\hline
42N & 1 & 0 & 1 & 0 & 1 & 0 & 1 & 1 & 0 & 1 & 1 & 1 & 1 & 0 & 0 & 0 & 1 & 1 & 0 & 0 & 1 & 1 & 0 & 1 & 0 & 0 & 0 & 1 & 0 & 1 & 0 & 1 & 0 & 0 & 0 & 0 & 0 & 1 & 1 & 1 \\
42P & 3 & 6 & 6 & 8 & 3 & 8 & 7 & 9 & 1 & 6 & 8 & 6 & 2 & 5 & 6 & 5 & 5 & 5 & 1 & 0 & 7 & 6 & 9 & 4 & 2 & 1 & 8 & 6 & 6 & 7 & 9 & 9 & 3 & 6 & 2 & 6 & 7 & 6 & 8 & 2 \\
\hline
43N & 1 & 0 & 1 & 1 & 1 & 0 & 1 & 0 & 0 & 1 & 0 & 0 & 1 & 0 & 1 & 0 & 0 & 1 & 0 & 1 & 1 & 1 & 0 & 1 & 1 & 0 & 0 & 1 & 0 & 0 & 0 & 1 & 0 & 0 & 0 & 1 & 1 & 1 & 1 & 1 \\
43P & 2 & 1 & 8 & 4 & 1 & 6 & 0 & 7 & 8 & 2 & 1 & 7 & 7 & 5 & 1 & 1 & 2 & 5 & 5 & 5 & 0 & 0 & 0 & 1 & 2 & 7 & 6 & 5 & 5 & 3 & 4 & 8 & 5 & 5 & 0 & 4 & 6 & 9 & 0 & 8 \\
\hline
44N & 0 & 1 & 0 & 1 & 1 & 1 & 1 & 1 & 1 & 1 & 1 & 1 & 1 & 1 & 1 & 0 & 1 & 1 & 1 & 0 & 1 & 1 & 1 & 1 & 0 & 1 & 1 & 1 & 0 & 0 & 1 & 1 & 1 & 1 & 1 & 0 & 1 & 0 & 1 & 1 \\
44P & 7 & 1 & 3 & 0 & 1 & 2 & 9 & 8 & 4 & 8 & 4 & 9 & 7 & 0 & 4 & 2 & 2 & 3 & 8 & 1 & 6 & 3 & 4 & 6 & 4 & 0 & 0 & 6 & 9 & 2 & 6 & 6 & 3 & 4 & 9 & 0 & 7 & 0 & 7 & 4 \\
\hline
45N & 1 & 0 & 1 & 0 & 1 & 0 & 0 & 1 & 1 & 0 & 1 & 0 & 1 & 1 & 1 & 0 & 1 & 0 & 0 & 0 & 1 & 0 & 1 & 0 & 1 & 1 & 0 & 0 & 0 & 1 & 1 & 0 & 0 & 1 & 0 & 0 & 0 & 0 & 1 & 0 \\
45P & 0 & 2 & 6 & 1 & 1 & 0 & 2 & 7 & 7 & 2 & 7 & 3 & 3 & 6 & 2 & 4 & 4 & 3 & 6 & 8 & 1 & 6 & 1 & 5 & 9 & 8 & 2 & 3 & 1 & 3 & 1 & 0 & 1 & 5 & 5 & 9 & 4 & 2 & 7 & 8 \\
\hline
46N & 1 & 0 & 1 & 1 & 1 & 1 & 1 & 0 & 0 & 0 & 0 & 0 & 1 & 0 & 1 & 0 & 1 & 0 & 0 & 1 & 0 & 0 & 0 & 1 & 0 & 0 & 0 & 1 & 0 & 1 & 0 & 1 & 0 & 0 & 1 & 0 & 0 & 0 & 1 & 0 \\
46P & 5 & 0 & 8 & 0 & 5 & 3 & 8 & 3 & 2 & 1 & 8 & 4 & 2 & 9 & 3 & 0 & 1 & 9 & 7 & 3 & 4 & 8 & 8 & 5 & 3 & 0 & 5 & 4 & 2 & 4 & 0 & 0 & 2 & 4 & 7 & 2 & 5 & 7 & 7 & 5 \\
\hline
47N & 0 & 0 & 1 & 0 & 0 & 1 & 0 & 0 & 1 & 0 & 0 & 1 & 1 & 0 & 0 & 1 & 1 & 1 & 1 & 1 & 1 & 0 & 0 & 1 & 0 & 0 & 1 & 0 & 1 & 0 & 0 & 1 & 1 & 1 & 1 & 0 & 0 & 1 & 1 & 1 \\
47P & 1 & 6 & 1 & 7 & 0 & 0 & 3 & 5 & 7 & 6 & 8 & 0 & 2 & 9 & 1 & 2 & 2 & 1 & 1 & 7 & 1 & 7 & 7 & 5 & 9 & 2 & 6 & 6 & 8 & 8 & 0 & 7 & 3 & 0 & 1 & 3 & 9 & 7 & 6 & 8 \\
\hline
48N & 1 & 0 & 0 & 0 & 0 & 0 & 0 & 0 & 1 & 0 & 1 & 0 & 0 & 1 & 1 & 1 & 0 & 0 & 1 & 1 & 1 & 0 & 1 & 1 & 1 & 0 & 0 & 1 & 1 & 1 & 1 & 1 & 1 & 1 & 0 & 1 & 1 & 1 & 1 & 1 \\
48P & 3 & 3 & 7 & 8 & 9 & 5 & 7 & 7 & 1 & 9 & 3 & 0 & 9 & 2 & 3 & 7 & 3 & 1 & 9 & 3 & 2 & 1 & 2 & 6 & 2 & 7 & 8 & 2 & 0 & 7 & 8 & 7 & 8 & 0 & 7 & 1 & 5 & 8 & 1 & 7 \\
\hline
49N & 0 & 0 & 1 & 1 & 1 & 1 & 1 & 1 & 0 & 1 & 0 & 1 & 1 & 0 & 1 & 1 & 1 & 0 & 1 & 0 & 1 & 0 & 1 & 1 & 0 & 0 & 0 & 1 & 1 & 0 & 1 & 1 & 0 & 1 & 1 & 0 & 1 & 0 & 1 & 0 \\
49P & 4 & 0 & 8 & 4 & 9 & 9 & 1 & 4 & 7 & 0 & 0 & 0 & 4 & 3 & 6 & 5 & 5 & 1 & 2 & 3 & 4 & 1 & 4 & 0 & 9 & 6 & 8 & 7 & 8 & 0 & 5 & 1 & 1 & 4 & 5 & 2 & 7 & 6 & 2 & 9 \\
\hline
50N & 0 & 1 & 1 & 0 & 1 & 1 & 0 & 0 & 1 & 0 & 1 & 1 & 1 & 1 & 1 & 0 & 0 & 1 & 1 & 0 & 1 & 0 & 1 & 1 & 0 & 1 & 0 & 1 & 0 & 0 & 1 & 0 & 0 & 1 & 1 & 0 & 0 & 0 & 0 & 1 \\
50P & 8 & 5 & 8 & 0 & 4 & 6 & 2 & 0 & 2 & 5 & 8 & 8 & 1 & 5 & 1 & 3 & 5 & 4 & 1 & 4 & 3 & 2 & 6 & 5 & 5 & 2 & 2 & 4 & 2 & 2 & 4 & 1 & 2 & 1 & 0 & 8 & 4 & 3 & 6 & 8 \\
\hline
51N & 0 & 1 & 0 & 0 & 0 & 0 & 0 & 0 & 1 & 1 & 1 & 0 & 0 & 0 & 0 & 1 & 0 & 0 & 1 & 1 & 0 & 1 & 0 & 0 & 1 & 0 & 1 & 1 & 0 & 0 & 1 & 1 & 0 & 1 & 1 & 1 & 1 & 1 & 1 & 1 \\
51P & 5 & 0 & 6 & 8 & 8 & 2 & 0 & 5 & 8 & 0 & 3 & 6 & 1 & 3 & 2 & 8 & 9 & 7 & 6 & 8 & 9 & 0 & 6 & 7 & 0 & 4 & 4 & 4 & 4 & 5 & 1 & 4 & 2 & 1 & 6 & 9 & 7 & 2 & 8 & 6 \\
\hline
52N & 1 & 0 & 1 & 1 & 1 & 1 & 1 & 0 & 0 & 0 & 1 & 0 & 0 & 0 & 1 & 0 & 0 & 0 & 1 & 0 & 1 & 0 & 0 & 1 & 1 & 1 & 1 & 1 & 0 & 0 & 0 & 0 & 0 & 1 & 0 & 1 & 0 & 1 & 0 & 1 \\
52P & 9 & 9 &   & 4 & 0 & 5 & 0 & 0 & 0 & 6 & 2 & 0 & 6 & 3 & 3 & 8 & 5 & 7 & 3 & 0 & 2 & 6 & 9 & 3 & 7 & 5 & 2 & 9 & 3 & 4 & 0 & 3 & 4 & 8 & 0 & 7 & 4 & 5 & 7 & 3 \\
\hline
53N & 1 & 1 & 0 & 0 & 0 & 0 & 0 & 1 & 0 & 1 & 0 & 1 & 1 & 1 & 1 & 1 & 0 & 1 & 0 & 1 & 0 & 1 & 1 & 1 & 0 & 1 & 1 & 0 & 0 & 0 & 0 & 1 & 1 & 0 & 0 & 0 & 0 & 1 & 0 & 1 \\
53P & 5 & 6 & 5 & 4 & 5 & 8 & 2 & 4 & 7 & 2 & 9 & 8 & 3 & 8 & 8 & 3 & 8 & 3 & 8 & 1 & 1 & 1 & 0 & 1 & 3 & 2 & 4 & 9 & 7 & 7 & 5 & 9 & 6 & 7 & 4 & 6 & 8 & 9 & 7 & 3 \\
\hline
54N & 0 & 0 & 0 & 1 & 0 & 1 & 0 & 1 & 1 & 1 & 1 & 1 & 1 & 0 & 1 & 0 & 0 & 1 & 0 & 1 & 1 & 0 & 1 & 1 & 1 & 1 & 0 & 0 & 0 & 1 & 1 & 1 & 0 & 1 & 0 & 0 & 0 & 1 & 0 & 0 \\
54P & 5 & 0 & 7 & 1 & 8 & 6 & 0 & 1 & 7 & 3 & 5 & 0 & 4 & 3 & 5 & 7 & 9 & 5 & 1 & 0 & 7 & 6 & 4 & 2 & 3 & 3 & 4 & 2 & 0 & 5 & 9 & 2 & 5 & 4 & 9 & 5 & 9 & 1 & 7 & 3 \\
\hline
55N &   & 0 & 0 & 1 & 1 & 1 & 0 & 0 & 0 & 0 & 1 & 0 & 0 & 0 & 1 & 1 & 1 & 1 & 1 & 1 & 1 & 0 & 0 & 0 & 0 & 1 & 1 & 1 & 1 & 1 & 0 & 1 & 1 & 1 & 0 & 1 & 1 & 0 & 0 & 0 \\
55P & 4 & 9 & 4 & 2 & 1 & 2 & 5 & 2 & 7 & 3 & 2 & 9 & 4 & 6 & 2 & 8 & 4 & 4 & 2 & 2 & 6 & 1 & 7 & 6 & 3 & 6 & 6 & 4 & 7 & 8 & 6 & 8 & 9 & 3 & 1 & 7 & 4 & 5 & 7 & 1 \\
\hline
56N & 1 & 0 & 1 & 1 & 1 & 0 & 0 & 1 & 0 & 1 & 0 & 1 & 1 & 1 & 1 & 1 & 0 & 1 & 0 & 1 & 1 & 0 & 0 & 1 & 0 & 0 & 1 & 0 & 1 & 0 & 1 & 0 & 0 & 1 & 0 & 0 & 1 & 1 & 0 & 0 \\
56P & 2 & 1 & 2 & 8 & 6 & 3 & 0 & 4 & 4 & 6 & 8 & 9 & 0 & 1 & 8 & 4 & 0 & 4 & 2 & 3 & 8 & 8 & 3 & 8 & 1 & 5 & 9 & 2 & 4 & 3 & 0 & 7 & 9 & 7 & 7 & 7 & 1 & 8 & 0 & 9 \\
\hline
57N & 0 & 1 & 0 & 1 & 0 & 1 & 0 & 1 & 0 & 0 & 0 & 0 & 0 & 1 & 1 & 0 & 0 & 0 & 0 & 1 & 0 & 0 & 0 & 0 & 0 & 0 & 1 & 0 & 1 & 1 & 0 & 0 & 1 & 1 & 0 & 0 & 0 & 1 & 0 & 0 \\
57P & 8 & 9 & 1 & 5 & 2 & 2 & 0 & 8 & 0 & 7 & 1 & 5 & 8 & 1 & 4 & 4 & 8 & 2 & 3 & 7 & 4 & 9 & 7 & 2 & 0 & 8 & 3 & 3 & 6 & 2 & 1 & 9 & 8 & 1 & 9 & 7 & 4 & 2 & 5 & 8 \\
\hline
58N & 0 & 0 & 1 & 0 & 1 & 1 & 0 & 0 & 0 & 0 & 0 & 1 & 0 & 0 & 1 & 1 & 0 & 1 & 0 & 1 & 1 & 1 & 1 & 0 & 1 & 0 & 1 & 1 & 0 & 0 & 1 & 1 & 0 & 1 & 1 & 0 & 0 & 0 & 0 & 0 \\
58P & 7 & 8 & 3 & 2 & 2 & 5 & 1 & 8 & 5 & 8 & 3 & 2 & 5 & 4 & 8 & 3 & 0 & 7 & 0 & 5 & 6 & 4 & 6 & 9 & 6 & 2 & 2 & 6 & 0 & 2 & 2 & 8 & 3 & 8 & 7 & 4 & 1 & 7 & 5 & 2 \\
\hline
59N & 1 & 0 & 0 & 1 & 1 & 1 & 1 & 0 & 0 & 0 & 1 & 0 & 0 & 1 & 0 & 1 & 1 & 0 & 0 & 1 & 1 & 0 & 1 & 0 & 0 & 0 & 1 & 1 & 1 & 0 & 0 & 1 & 0 & 0 & 0 & 0 & 1 & 1 & 0 & 1 \\
59P & 5 & 4 & 4 & 4 & 2 & 6 & 0 & 6 & 6 & 7 & 3 & 6 & 6 & 3 & 8 & 2 & 3 & 9 & 5 & 0 & 3 & 4 & 4 & 8 & 1 & 8 & 9 & 8 & 0 & 3 & 0 & 1 & 5 & 4 & 7 & 9 & 3 & 7 & 9 & 5 \\
\hline
60N & 1 & 0 & 1 & 0 & 1 & 0 & 0 & 1 & 0 & 1 & 1 & 1 & 0 & 0 & 0 & 1 & 1 & 1 & 0 & 1 & 0 & 0 & 0 & 0 & 0 & 0 & 1 & 1 & 1 & 0 & 0 & 1 & 0 & 0 & 0 & 1 & 0 & 1 & 0 & 0 \\
60P & 9 & 3 & 7 & 8 & 4 & 6 & 3 & 8 & 6 & 2 & 9 & 9 & 0 & 3 & 1 & 8 & 2 & 6 & 3 & 4 & 3 & 1 & 7 & 6 & 6 & 0 & 1 & 5 & 1 & 3 & 0 & 9 & 4 & 8 & 1 & 9 & 2 & 6 & 2 & 9 \\
\hline
61N & 0 & 0 & 1 & 1 & 1 & 0 & 0 & 0 & 0 & 0 & 1 & 1 & 1 & 1 & 1 & 0 & 1 & 1 & 1 & 0 & 0 & 1 & 1 & 1 & 0 & 0 & 0 & 1 & 1 & 0 & 0 & 0 & 1 & 0 & 0 & 1 & 0 & 0 & 1 & 0 \\
61P & 9 & 5 & 2 & 7 & 3 & 4 & 4 & 3 & 7 & 2 & 3 & 9 & 3 & 0 & 1 & 5 & 2 & 8 & 1 & 0 & 6 & 0 & 0 & 7 & 1 & 9 & 4 & 3 & 0 & 4 & 7 & 4 & 2 & 4 & 8 & 5 & 6 & 7 & 5 & 7 \\
\hline
62N & 0 & 1 & 1 & 0 & 1 & 0 & 1 & 0 & 1 & 1 & 1 & 0 & 0 & 1 & 1 & 0 & 0 & 0 & 0 & 0 & 1 & 0 & 0 & 0 & 1 & 1 & 0 & 0 & 1 & 0 & 1 & 0 & 1 & 0 & 0 & 0 & 1 & 0 & 1 & 1 \\
62P & 5 & 8 & 8 & 6 & 1 & 2 & 9 & 7 & 4 & 5 & 2 & 4 & 0 & 4 & 7 & 3 & 5 & 2 & 8 & 6 & 5 & 6 & 5 & 2 & 6 & 2 & 2 & 4 & 2 & 3 & 2 & 5 & 9 & 5 & 1 & 5 & 0 & 7 & 8 & 2 \\
\hline
63N & 1 & 0 & 0 & 1 & 0 & 1 & 0 & 1 & 1 & 1 & 1 & 1 & 0 & 0 & 0 & 0 & 0 & 0 & 0 & 0 & 1 & 0 & 1 & 1 & 0 & 0 & 1 & 1 & 1 & 0 & 0 & 0 & 1 & 0 & 0 & 0 & 0 & 1 & 0 & 1 \\
63P & 1 & 6 & 5 & 2 & 5 & 3 & 9 & 1 & 0 & 7 & 6 & 5 & 3 & 0 & 7 & 7 & 1 & 1 & 6 & 6 & 0 & 6 & 3 & 2 & 3 & 8 & 0 & 4 & 7 & 0 & 5 & 7 & 1 & 6 & 1 & 9 & 6 & 3 & 1 & 3 \\
\hline
64N & 1 & 1 & 0 & 0 & 0 & 0 & 1 & 1 & 1 & 0 & 1 & 1 & 0 & 1 & 1 & 0 & 1 & 1 & 1 & 0 & 1 & 1 & 0 & 1 & 0 & 0 & 0 & 1 & 1 & 0 & 1 & 1 & 1 & 1 & 0 & 1 & 1 & 1 & 1 & 0 \\
64P & 0 & 5 & 8 &   & 0 & 2 & 3 & 0 & 2 & 1 & 3 & 3 & 3 & 9 & 5 & 1 & 4 & 6 & 5 & 3 & 1 & 2 & 2 & 7 & 5 & 9 & 7 & 6 & 0 & 4 & 0 & 7 & 5 & 9 & 0 & 3 & 9 & 9 & 2 & 4 \\
\hline
65N & 0 & 1 & 0 & 1 & 1 & 1 & 0 & 0 & 0 & 1 & 1 & 0 & 0 & 1 & 0 & 0 & 0 & 1 & 1 & 1 & 1 & 1 & 0 & 1 & 1 & 0 & 0 & 1 & 0 & 0 & 0 & 1 & 1 & 1 & 1 & 0 & 1 & 0 & 0 & 0 \\
65P & 3 & 8 & 3 & 6 & 5 & 9 & 8 & 7 & 3 & 7 & 8 & 9 & 5 & 2 & 0 & 2 & 0 & 0 & 1 & 9 & 7 & 8 & 4 & 9 & 3 & 7 & 5 & 0 & 5 & 0 & 8 & 2 & 3 & 3 & 6 & 1 & 5 & 2 & 9 & 5 \\
\hline
66N & 1 & 0 & 0 & 1 & 1 & 0 & 1 & 1 & 0 & 0 & 0 & 1 & 1 & 1 & 1 & 0 & 0 & 1 & 1 & 0 & 0 & 0 & 1 & 1 & 0 & 1 & 0 & 1 & 1 & 0 & 1 & 0 & 1 & 1 & 1 & 1 & 1 & 0 & 1 & 0 \\
66P & 7 & 7 & 8 & 4 & 9 & 2 & 1 & 5 & 0 & 6 & 3 & 7 & 9 & 8 & 5 & 5 & 7 & 8 & 1 & 3 & 9 & 3 & 8 & 4 & 2 & 5 & 0 & 9 & 8 & 2 & 1 & 6 & 9 & 4 & 9 & 3 & 8 & 9 & 4 & 9 \\
\hline
67N & 0 & 0 & 1 & 0 & 0 & 0 & 1 & 1 & 0 & 1 & 0 & 1 & 1 & 0 & 1 & 1 & 1 & 1 & 1 & 1 & 0 & 1 & 1 & 1 & 0 & 0 & 0 & 0 & 1 & 1 & 0 & 0 & 1 & 0 & 1 & 0 & 0 & 1 & 0 & 0 \\
67P & 2 & 4 & 2 & 6 & 3 & 6 & 5 & 8 & 7 & 2 & 0 & 5 & 1 & 2 & 5 & 3 & 3 & 0 & 1 & 0 & 0 & 9 & 9 & 9 & 5 & 7 & 6 & 9 & 3 & 3 & 9 & 1 & 8 & 0 & 7 & 5 & 8 & 6 & 1 & 2 \\
\hline
68N & 1 & 0 & 1 & 0 & 0 & 0 & 1 & 0 & 1 & 0 & 1 & 1 & 0 & 1 & 1 & 0 & 0 & 0 & 0 & 1 & 0 & 1 & 0 & 0 & 0 & 0 & 1 & 1 & 0 & 1 & 1 & 1 & 0 & 0 & 1 & 0 & 1 & 0 & 0 & 1 \\
68P & 1 & 0 & 9 & 6 & 0 & 8 & 0 & 4 & 7 & 1 & 7 & 1 & 3 & 1 & 2 & 2 & 1 & 0 & 1 & 7 & 8 & 9 & 7 & 7 & 6 & 4 & 4 & 1 & 7 & 9 & 6 & 3 & 0 & 0 & 3 & 1 & 4 & 4 & 6 & 0 \\
\hline
69N & 1 & 0 & 0 & 1 & 1 & 1 & 0 & 1 & 1 & 0 & 0 & 0 & 1 & 1 & 0 & 0 & 0 & 0 & 0 & 1 & 0 & 0 & 0 & 0 & 1 & 1 & 0 & 1 & 0 & 0 & 1 & 1 & 0 & 1 & 0 & 0 & 0 & 0 & 0 & 1 \\
69P & 0 & 6 & 7 & 3 & 8 & 7 & 8 & 1 & 9 & 4 & 4 & 5 & 0 & 0 & 4 & 2 & 5 & 8 & 2 & 9 & 4 & 2 & 1 & 7 & 2 & 3 & 1 & 3 & 9 & 8 & 6 & 4 & 3 & 3 & 2 & 7 & 4 & 9 & 4 & 0 \\
\hline
70N & 1 & 0 & 0 & 1 & 0 & 0 & 1 & 1 & 1 & 1 & 1 & 1 & 1 & 0 & 0 & 0 & 0 & 1 & 1 & 0 & 0 & 0 & 0 & 0 & 0 & 1 & 0 & 1 & 1 & 1 & 0 & 0 & 0 & 0 & 0 & 0 & 0 & 0 & 0 & 1 \\
70P & 8 & 8 & 7 & 0 & 0 & 9 & 7 & 7 & 3 & 7 & 1 & 7 & 4 & 1 & 9 & 8 & 4 & 6 & 2 & 4 & 4 & 5 & 0 & 8 & 5 & 7 & 2 & 5 & 8 & 6 & 0 & 1 & 0 & 7 & 6 & 4 & 8 & 0 & 4 & 9 \\
\hline
71N & 0 & 0 & 1 & 1 & 1 & 0 & 1 & 0 & 1 & 0 & 0 & 1 & 0 & 1 & 0 & 1 & 0 & 0 & 1 & 0 & 1 & 0 & 1 & 1 & 1 & 0 & 1 & 1 & 0 & 1 & 1 & 0 & 1 & 0 & 1 & 1 & 1 & 0 & 1 & 0 \\
71P & 8 & 2 & 9 & 8 & 5 & 7 & 0 & 5 & 6 & 4 & 1 & 4 & 3 & 8 & 3 & 0 & 6 & 5 & 4 & 1 & 7 & 7 & 6 & 5 & 8 & 7 & 6 & 2 & 2 & 1 & 9 & 1 & 3 & 8 & 1 & 3 & 9 & 3 & 1 & 8 \\
\hline
72N & 0 & 1 & 0 & 1 & 0 & 1 & 0 & 1 & 1 & 1 & 1 & 1 & 1 & 0 & 1 & 1 & 1 & 0 & 0 & 1 & 1 & 0 & 1 & 0 & 1 & 0 & 1 & 0 & 1 & 1 & 0 & 1 & 0 & 1 & 1 & 1 & 1 & 0 & 0 & 0 \\
72P & 2 & 6 & 4 & 8 & 8 & 6 & 0 & 0 & 7 & 3 & 8 & 2 & 9 & 0 & 9 & 8 & 7 & 5 & 2 & 0 & 7 & 4 & 8 & 7 & 6 & 7 & 4 & 5 & 7 & 4 & 0 & 1 & 8 & 5 & 0 & 7 & 4 & 4 & 3 & 0 \\
\hline
73N & 1 & 1 & 1 & 1 & 0 & 0 & 1 & 0 & 1 & 1 & 1 & 0 & 0 & 1 & 1 & 1 & 1 & 1 & 0 & 0 & 0 & 1 & 0 & 1 & 0 & 0 & 1 & 1 & 1 & 0 & 1 & 0 & 0 & 1 & 0 & 0 & 1 & 0 & 1 & 1 \\
73P & 4 & 9 & 2 & 5 & 7 & 4 & 8 & 4 & 5 & 1 & 5 & 2 & 0 & 7 & 8 & 0 & 9 & 0 & 5 & 7 & 8 & 8 & 5 & 0 & 8 & 1 & 3 & 8 & 6 & 1 & 3 & 1 & 9 & 5 & 5 & 0 & 8 & 0 & 3 & 5 \\
\hline
74N & 1 & 0 & 1 & 1 & 0 & 0 & 0 & 0 & 0 & 0 & 1 & 1 & 0 & 0 & 0 & 0 & 1 & 1 & 0 & 1 & 0 & 0 & 1 & 1 & 0 & 0 & 0 & 1 & 1 & 0 & 0 & 1 & 0 & 0 & 0 & 1 & 1 & 0 & 0 & 1 \\
74P & 3 & 0 & 0 & 2 & 9 & 3 & 0 & 3 & 5 & 4 & 6 & 5 & 0 & 2 & 5 & 7 & 4 & 9 & 8 & 6 & 1 & 4 & 0 & 0 & 4 & 6 & 8 & 1 & 1 & 5 & 8 & 6 & 9 & 9 & 9 & 7 & 9 & 4 & 2 & 1 \\
\hline
75N & 0 & 1 & 0 & 1 & 0 & 0 & 0 & 0 & 1 & 1 & 0 & 0 & 0 & 1 & 0 & 1 & 0 & 0 & 1 & 1 & 0 & 1 & 1 & 0 & 0 & 0 & 1 & 1 & 1 & 1 & 1 & 1 & 1 & 1 & 0 & 1 & 0 & 1 & 1 & 0 \\
75P & 1 & 7 & 9 & 1 & 3 & 7 & 6 & 8 & 6 & 5 & 7 & 1 & 3 & 4 & 0 & 8 & 6 & 9 & 1 & 2 & 4 & 3 & 3 & 3 & 2 & 0 & 0 & 9 & 9 & 3 & 3 & 0 & 3 & 8 & 5 & 5 & 7 & 3 & 8 & 3 \\
\hline
76N & 1 & 0 & 0 & 0 & 1 & 0 & 1 & 1 & 1 & 0 & 0 & 0 & 1 & 1 & 0 & 0 & 1 & 1 & 0 & 0 & 0 & 1 & 1 & 1 & 1 & 0 & 1 & 1 & 0 & 0 & 1 & 0 & 0 & 1 & 1 & 1 & 1 & 0 & 0 & 0 \\
76P & 3 & 5 & 8 & 6 & 7 & 7 & 9 & 7 & 1 & 6 & 3 & 1 & 8 & 2 & 0 & 7 & 3 & 1 & 6 & 4 & 0 & 8 & 1 & 7 & 7 & 4 & 9 & 6 & 3 & 4 & 5 & 5 & 2 & 1 & 6 & 4 & 8 & 0 & 2 & 6 \\
\hline
77N & 0 & 0 & 0 & 1 & 1 & 0 & 0 & 1 & 0 & 1 & 1 & 0 & 0 & 0 & 0 & 0 & 0 & 0 & 0 & 0 & 1 & 0 & 1 & 1 & 1 & 1 & 1 & 0 & 1 & 1 & 0 & 0 & 0 & 1 & 0 & 1 & 0 & 1 & 1 & 1 \\
77P & 9 & 3 & 3 & 2 & 4 & 9 & 7 & 1 & 2 & 9 & 2 & 2 & 7 & 1 & 6 & 7 & 8 & 9 & 0 & 7 & 1 & 9 & 3 & 3 & 2 & 1 & 8 & 0 & 3 & 6 & 5 & 1 & 2 & 6 & 3 & 1 & 0 & 5 & 5 & 8 \\
\hline
78N & 0 & 1 & 0 & 1 & 1 & 1 & 1 & 0 & 1 & 1 & 0 & 1 & 1 & 0 & 0 & 1 & 0 & 1 & 0 & 0 & 0 & 1 & 0 & 1 & 0 & 1 & 1 & 1 & 1 & 0 & 1 & 0 & 0 & 1 & 1 & 1 & 1 & 1 & 0 & 0 \\
78P & 7 & 7 & 2 & 3 & 4 & 3 & 8 & 8 & 9 & 2 & 3 & 8 & 0 & 0 & 0 & 0 & 2 & 4 & 7 & 6 & 3 & 6 & 6 & 1 & 1 & 7 & 3 & 8 & 8 & 8 & 2 & 0 & 8 & 6 & 5 & 3 & 0 & 3 & 0 & 2 \\
\hline
79N & 1 & 0 & 1 & 0 & 0 & 0 & 1 & 1 & 1 & 1 & 1 & 1 & 1 & 0 & 1 & 0 & 0 & 1 & 1 & 0 & 1 & 0 & 1 & 1 & 0 & 0 & 0 & 0 & 1 & 0 & 0 & 0 & 1 & 0 & 1 & 0 & 0 & 0 & 0 & 0 \\
79P & 1 & 4 & 9 & 4 & 5 & 1 & 1 & 8 & 1 & 2 & 8 & 8 & 9 & 8 & 5 & 5 & 6 & 7 & 2 & 5 & 2 & 3 & 9 & 3 & 2 & 7 & 5 & 2 & 4 & 3 & 8 & 1 & 9 & 8 & 9 & 3 & 2 & 6 & 1 & 2 \\
\hline
80N & 0 & 0 & 0 & 1 & 0 & 1 & 0 & 0 & 0 & 0 & 1 & 1 & 0 & 1 & 0 & 0 & 0 & 1 & 0 & 0 & 0 & 1 & 0 & 0 & 0 & 1 & 1 & 0 & 0 & 1 & 0 & 0 & 0 & 1 & 0 & 0 & 1 & 0 & 0 & 1 \\
80P & 3 & 2 & 5 & 2 & 4 & 0 & 0 & 6 & 0 & 8 & 5 & 6 & 6 & 8 & 2 & 2 & 9 & 9 & 2 & 1 & 6 & 8 & 2 & 0 & 9 & 9 & 3 & 4 & 6 & 2 & 6 & 8 & 8 & 9 & 6 & 8 & 4 & 4 & 0 & 5 \\
\hline
81N & 0 & 1 & 1 & 1 & 0 & 1 & 0 & 1 & 0 & 1 & 1 & 0 & 1 & 0 & 1 & 0 & 0 & 0 & 0 & 1 & 0 & 1 & 1 & 0 & 0 & 1 & 0 & 1 & 1 & 0 & 0 & 1 & 1 & 1 & 1 & 0 & 1 & 0 & 1 & 0 \\
81P & 6 & 4 & 2 & 3 & 0 & 7 & 7 & 5 & 0 & 8 & 9 & 3 & 7 & 3 & 5 & 9 & 1 & 6 & 9 & 8 & 0 & 4 & 4 & 5 & 6 & 4 & 7 & 7 & 3 & 1 & 1 & 7 & 5 & 7 & 8 & 4 & 1 & 3 & 3 & 9 \\
\hline
82N & 1 & 1 & 0 & 0 & 0 & 0 & 0 & 1 & 0 & 1 & 1 & 1 & 0 & 1 & 1 & 1 & 1 & 1 & 0 & 1 & 1 & 1 & 1 & 1 & 1 & 1 & 1 & 1 & 0 & 1 & 0 & 0 & 1 & 1 & 0 & 1 & 1 & 0 & 1 & 1 \\
82P & 2 & 8 & 1 & 7 & 1 & 2 & 6 & 3 & 9 & 1 & 1 & 6 & 4 & 6 & 1 & 9 & 9 & 7 & 9 & 7 & 6 & 7 & 0 & 8 & 4 & 2 & 7 & 3 & 4 & 7 & 8 & 2 & 7 & 5 & 8 & 4 & 1 & 4 & 7 & 1 \\
\hline
83N & 0 & 0 & 1 & 1 & 0 & 1 & 0 & 0 & 0 & 1 & 0 & 0 & 1 & 0 & 0 & 0 & 0 & 1 & 0 & 1 & 0 & 1 & 1 & 1 & 0 & 1 & 1 & 1 & 0 & 0 & 1 & 1 & 1 & 1 & 0 & 0 & 1 & 0 & 0 & 1 \\
83P & 0 & 9 & 1 & 7 & 3 & 9 & 5 & 8 & 2 & 4 & 9 & 9 & 8 & 2 & 1 & 2 & 2 & 0 & 5 & 6 & 1 & 4 & 5 & 2 & 8 & 6 & 6 & 5 & 2 & 3 & 9 & 9 & 3 & 6 & 8 & 5 & 8 & 9 & 7 & 4 \\
\hline
84N & 1 & 1 & 1 & 0 & 0 & 1 & 0 & 1 & 1 & 0 & 1 & 0 & 0 & 0 & 1 & 1 & 1 & 1 & 1 & 0 & 0 & 0 & 0 & 1 & 0 & 1 & 1 & 1 & 0 & 1 & 0 & 0 & 1 & 1 & 0 & 1 & 0 & 0 & 0 & 1 \\
84P & 0 & 4 & 3 & 9 & 1 & 0 & 4 & 0 & 1 & 3 & 5 & 1 & 2 & 4 & 2 & 5 & 0 & 1 & 7 & 7 & 5 & 6 & 2 & 0 & 4 & 3 & 9 & 4 & 8 & 6 & 0 & 3 & 5 & 6 & 0 & 9 & 0 & 2 & 8 & 5 \\
\hline
85N & 1 & 0 & 0 & 0 & 1 & 0 & 1 & 0 & 0 & 0 & 1 & 1 & 0 & 0 & 0 & 0 & 1 & 0 & 1 & 1 & 1 & 0 & 0 & 0 & 1 & 1 & 0 & 1 & 1 & 1 & 1 & 0 & 0 & 0 & 1 & 0 & 0 & 0 & 1 & 0 \\
85P & 8 & 1 & 6 & 6 & 6 & 9 & 9 & 7 & 8 & 4 & 1 & 5 & 8 & 4 & 6 & 8 & 3 & 7 & 3 & 0 & 8 & 0 & 8 & 3 & 6 & 1 & 4 & 3 & 5 & 4 & 7 & 7 & 7 & 7 & 1 & 4 & 2 & 8 & 9 & 3 \\
\hline
86N & 0 & 0 & 1 & 0 & 1 & 1 & 0 & 1 & 0 & 1 & 0 & 0 & 0 & 1 & 0 & 1 & 1 & 0 & 0 & 0 & 0 & 0 & 0 & 1 & 0 & 1 & 0 & 1 & 1 & 0 & 0 & 0 & 0 & 0 & 1 & 0 & 0 & 0 & 1 & 1 \\
86P & 0 & 7 & 2 & 4 & 8 & 4 & 1 & 8 & 1 & 8 & 3 & 6 & 7 & 0 & 7 & 6 & 7 & 0 & 9 & 7 & 9 & 6 & 5 & 4 & 2 & 1 & 9 & 3 & 4 & 8 & 8 & 6 & 9 & 8 & 9 & 2 & 0 & 6 & 2 & 2 \\
\hline
87N & 1 & 1 & 1 & 0 & 1 & 0 & 0 & 0 & 0 & 1 & 1 & 1 & 1 & 0 & 0 & 1 & 1 & 0 & 0 & 0 & 1 & 1 & 1 & 0 & 1 & 1 & 1 & 1 & 1 & 1 & 0 & 0 & 1 & 1 & 0 & 0 & 0 & 1 & 0 & 0 \\
87P & 0 & 0 & 2 & 3 & 3 & 0 & 6 & 0 & 6 & 1 & 1 & 7 & 6 & 2 & 4 & 1 & 4 & 2 & 1 & 9 & 9 & 6 & 2 & 7 & 5 & 5 & 6 & 7 & 4 & 2 & 0 & 0 & 2 & 2 & 2 & 2 & 8 & 4 & 1 & 1 \\
\hline
88N & 0 & 1 & 0 & 1 & 0 & 0 & 1 & 0 & 1 & 0 & 0 & 0 & 0 & 0 & 0 & 0 & 0 & 0 & 1 & 1 & 1 & 0 & 1 & 0 & 1 & 0 & 1 & 0 & 1 & 0 & 1 & 1 & 1 & 0 & 0 & 1 & 1 & 0 & 0 & 1 \\
88P & 9 & 6 & 2 & 6 & 9 & 7 & 7 & 9 & 9 & 1 & 6 & 1 & 6 & 8 & 3 & 5 & 4 & 1 & 2 & 2 & 0 & 1 & 5 & 4 & 4 & 5 & 8 & 9 & 9 & 2 & 0 & 3 & 8 & 4 & 0 & 0 & 3 & 3 & 7 & 5 \\
\hline
89N & 0 & 0 & 1 & 1 & 1 & 0 & 1 & 0 & 1 & 0 & 0 & 1 & 1 & 0 & 0 & 0 & 0 & 0 & 1 & 0 & 1 & 1 & 1 & 1 & 1 & 0 & 1 & 1 & 0 & 0 & 1 & 1 & 0 & 0 & 0 & 0 & 0 & 1 & 1 & 1 \\
89P & 4 & 6 & 4 & 3 & 5 & 5 & 4 & 0 & 2 & 1 & 6 & 7 & 4 & 8 & 9 & 3 & 7 & 9 & 0 & 9 & 6 & 5 & 1 & 6 & 2 & 8 & 4 & 4 & 8 & 7 & 1 & 5 & 4 & 6 & 2 & 6 & 2 & 8 & 6 & 5 \\
\hline
90N & 0 & 1 & 1 & 0 & 1 & 0 & 1 & 0 & 1 & 0 & 1 & 1 & 0 & 0 & 0 & 0 & 0 & 1 & 1 & 0 & 0 & 0 & 1 & 1 & 0 & 1 & 0 & 1 & 1 & 0 & 0 & 0 & 0 & 1 & 1 & 0 & 0 & 0 & 0 & 1 \\
90P & 7 & 8 & 2 & 2 & 0 & 6 & 0 & 2 & 1 & 8 & 3 & 7 & 8 & 6 & 3 & 1 & 1 & 3 & 3 & 4 & 9 & 1 & 3 & 6 & 2 & 1 & 2 & 3 & 1 & 8 & 6 & 9 & 2 & 3 & 9 & 0 & 0 & 3 & 9 & 4 \\
\hline
91N & 1 & 0 & 1 & 1 & 1 & 1 & 0 & 1 & 0 & 1 & 0 & 1 & 1 & 1 & 0 & 1 & 1 & 1 & 1 & 1 & 0 & 1 & 0 & 1 & 0 & 1 & 0 & 0 & 1 & 1 & 0 & 1 & 0 & 1 & 1 & 0 & 1 & 1 & 0 & 0 \\
91P & 6 & 5 & 1 & 5 & 8 & 5 & 3 & 9 & 8 & 0 & 8 & 5 & 4 & 9 & 9 & 8 & 4 & 4 & 4 & 8 & 5 & 7 & 8 & 5 & 9 & 7 & 1 & 8 & 8 & 6 & 4 & 5 & 4 & 4 & 2 & 9 & 6 & 0 & 4 & 6 \\
\hline
92N & 1 & 1 & 0 & 1 & 0 & 1 & 0 & 1 & 0 & 1 & 1 & 1 & 1 & 1 & 0 & 1 & 0 & 0 & 0 & 1 & 1 & 0 & 1 & 0 & 1 & 1 & 1 & 1 & 1 & 1 & 0 & 0 & 1 & 1 & 1 & 0 & 1 & 1 & 1 & 1 \\
92P & 7 & 5 & 8 & 6 & 2 & 6 & 2 & 7 & 2 & 9 & 2 & 3 & 9 & 0 & 8 & 7 & 6 & 0 & 2 & 5 & 6 & 5 & 5 & 0 & 8 & 1 & 9 & 8 & 9 & 1 & 4 & 1 & 5 & 6 & 6 & 5 & 4 & 5 & 2 & 6 \\
\hline
93N & 1 & 1 & 0 & 1 & 0 & 0 & 0 & 1 & 1 & 0 & 0 & 0 & 1 & 0 & 0 & 0 & 0 & 0 & 1 & 0 & 0 & 1 & 0 & 1 & 1 & 1 & 0 & 0 & 0 & 1 & 0 & 1 & 0 & 0 & 1 & 1 & 1 & 1 & 0 & 0 \\
93P & 8 & 5 & 6 & 1 & 8 & 2 & 8 & 6 & 9 & 6 & 5 & 2 & 0 & 9 & 8 & 1 & 7 & 1 & 6 & 6 & 4 & 7 & 8 & 7 & 7 & 9 & 3 & 4 & 1 & 6 & 0 & 6 & 6 & 7 & 7 & 9 & 3 & 1 & 5 & 0 \\
\hline
94N & 1 & 0 & 1 & 1 & 0 & 0 & 0 & 0 & 0 & 0 & 0 & 1 & 1 & 1 & 1 & 1 & 1 & 0 & 1 & 0 & 0 & 1 & 1 & 1 & 1 & 1 & 0 & 1 & 1 & 0 & 0 & 1 & 1 & 1 & 1 & 0 & 0 & 0 & 0 & 1 \\
94P & 4 & 1 & 4 & 4 & 6 & 6 & 3 & 3 & 1 & 8 & 8 & 3 & 8 & 8 & 8 & 1 & 5 & 5 & 5 & 2 & 6 & 1 & 2 & 7 & 2 & 5 & 8 & 9 & 7 & 3 & 8 & 2 & 8 & 4 & 6 & 7 & 8 & 6 & 6 & 8 \\
\hline
95N & 0 & 1 & 0 & 1 & 0 & 1 & 1 & 0 & 1 & 1 & 1 & 1 & 0 & 1 & 1 & 0 & 0 & 0 & 0 & 0 & 1 & 0 & 1 & 1 & 0 & 0 & 1 & 1 & 0 & 0 & 1 & 1 & 0 & 0 & 1 & 1 & 0 & 0 & 1 & 0 \\
95P & 3 & 4 & 9 & 8 & 5 & 7 & 1 & 9 & 1 & 9 & 3 & 1 & 3 & 6 & 6 & 4 & 3 & 0 & 7 & 0 & 4 & 2 & 1 & 2 & 9 & 4 & 0 & 2 & 7 & 6 & 7 & 7 & 5 & 9 & 7 & 7 & 2 & 0 & 4 & 5 \\
\hline
96N & 0 & 0 & 0 & 1 & 1 & 0 & 1 & 1 & 0 & 1 & 0 & 0 & 1 & 1 & 1 & 1 & 0 & 0 & 0 & 0 & 0 & 1 & 0 & 1 & 0 & 1 & 1 & 1 & 0 & 0 & 1 & 0 & 0 & 0 & 0 & 0 & 0 & 0 & 1 & 1 \\
96P & 9 & 1 & 8 & 9 & 6 & 6 & 5 & 0 & 7 & 7 & 6 & 6 & 2 & 6 & 4 & 3 & 0 & 3 & 2 & 2 & 6 & 6 & 2 & 7 & 1 & 6 & 9 & 6 & 9 & 5 & 5 & 4 & 8 & 7 & 3 & 6 & 6 & 7 & 5 & 8 \\
\hline
97N & 1 & 1 & 0 & 0 & 0 & 0 & 0 & 1 & 1 & 1 & 0 & 1 & 1 & 1 & 1 & 1 & 1 & 0 & 0 & 0 & 1 & 1 & 1 & 1 & 1 & 1 & 0 & 1 & 1 & 1 & 0 & 0 & 1 & 1 & 1 & 1 & 1 & 1 & 0 & 0 \\
97P & 5 & 4 & 6 & 3 & 3 & 2 & 6 & 8 & 4 & 4 & 2 & 5 & 1 & 5 & 4 & 7 & 9 & 3 & 1 & 3 & 4 & 7 & 0 & 7 & 7 & 4 & 3 & 8 & 1 & 9 & 1 & 3 & 8 & 6 & 9 & 6 & 6 & 8 & 4 & 8 \\
\hline
98N & 0 & 0 & 0 & 1 & 0 & 1 & 1 & 0 & 1 & 1 & 1 & 1 & 1 & 1 & 1 & 0 & 0 & 1 & 1 & 1 & 0 & 1 & 0 & 1 & 1 & 0 & 0 & 0 & 0 & 0 & 0 & 0 & 0 & 0 & 1 & 0 & 0 & 1 & 0 & 0 \\
98P & 4 & 0 & 0 & 8 & 1 & 5 & 0 & 4 & 0 & 4 & 9 & 7 & 0 & 3 & 3 & 9 & 0 & 3 & 2 & 2 & 6 & 8 & 4 & 3 & 1 & 4 & 2 & 6 & 9 & 9 & 8 & 7 & 3 & 1 & 7 & 7 & 5 & 1 & 2 & 8 \\
\hline
99N & 0 & 1 & 0 & 1 & 1 & 0 & 1 & 1 & 1 & 1 & 1 & 1 & 0 & 0 & 0 & 1 & 0 & 1 & 1 & 1 & 0 & 1 & 1 & 0 & 1 & 0 & 1 & 0 & 1 & 0 & 1 & 0 & 0 & 1 & 1 & 1 & 0 & 0 & 0 & 1 \\
99P & 3 & 9 & 3 & 5 & 6 & 7 & 0 & 9 & 5 & 0 & 4 & 1 & 1 & 2 & 1 & 4 & 1 & 2 & 3 & 7 & 9 & 4 & 7 & 3 & 0 & 6 & 6 & 8 & 1 & 1 & 4 & 8 & 4 & 0 & 1 & 6 & 0 & 8 & 6 & 0 \\
\hline
100N & 0 & 0 & 1 & 1 & 0 & 1 & 0 & 1 & 0 & 1 & 0 & 1 & 0 & 1 & 1 & 1 & 1 & 1 & 1 & 0 & 1 & 0 & 1 & 1 & 0 & 1 & 1 & 0 & 0 & 1 & 1 & 1 & 0 & 0 & 0 & 0 & 0 & 0 & 1 & 1 \\
100P & 0 & 4 & 2 & 8 & 1 & 4 & 6 & 2 & 2 & 7 & 7 & 5 & 8 & 4 & 1 & 6 & 4 & 6 & 5 & 0 & 8 & 5 & 7 & 5 & 6 & 8 & 3 & 4 & 0 & 1 & 2 & 6 & 8 & 9 & 2 & 9 & 9 & 5 & 9 & 3 \\
\hline
101N & 1 & 1 & 0 & 1 & 1 & 1 & 1 & 0 & 1 & 1 & 0 & 1 & 1 & 1 & 0 & 0 & 0 & 1 & 1 & 1 & 0 & 1 & 1 & 0 & 1 & 1 & 1 & 1 & 0 & 0 & 0 & 1 & 1 & 0 & 0 & 0 & 0 & 0 & 0 & 0 \\
101P & 8 & 7 & 0 & 5 & 3 & 5 & 4 & 1 & 8 & 1 & 5 & 9 & 2 & 7 & 2 & 7 & 2 & 8 & 1 & 2 & 2 & 3 & 7 & 7 & 3 & 0 & 3 & 1 & 5 & 2 & 1 & 5 & 2 & 2 & 6 & 0 & 2 & 2 & 0 & 3 \\
\hline
102N & 1 & 1 & 1 & 1 & 0 & 1 & 1 & 0 & 0 & 1 & 0 & 0 & 0 & 1 & 0 & 0 & 0 & 1 & 1 & 1 & 1 & 0 & 0 & 0 & 1 & 0 & 0 & 0 & 1 & 1 & 1 & 0 & 0 & 1 & 0 & 0 & 1 & 0 & 0 & 0 \\
102P & 3 & 3 & 4 & 5 & 4 & 0 & 8 & 3 & 5 & 0 & 3 & 6 & 2 & 3 & 8 & 6 & 7 & 8 & 5 & 8 & 5 & 5 & 6 & 9 & 0 & 3 & 8 & 0 & 1 & 5 & 4 & 6 & 3 & 4 & 2 & 6 & 7 & 9 & 2 & 5 \\
\hline
103N & 1 & 0 & 0 & 0 & 0 & 0 & 1 & 0 & 0 & 0 & 1 & 1 & 0 & 0 & 1 & 1 & 1 & 1 & 0 & 0 & 1 & 0 & 1 & 1 & 1 & 0 & 0 & 0 & 0 & 1 & 0 & 0 & 1 & 0 & 1 & 0 & 1 & 0 & 1 & 1 \\
103P & 5 & 6 & 9 & 8 & 5 & 8 & 0 & 5 & 6 & 6 & 8 & 9 & 9 & 2 & 3 & 9 & 7 & 6 & 7 & 1 & 7 & 8 & 4 & 7 & 0 & 2 & 1 & 0 & 4 & 4 & 8 & 1 & 6 & 0 & 7 & 8 & 6 & 7 & 9 & 9 \\
\hline
104N & 0 & 0 & 1 & 1 & 0 &   & 0 & 1 & 0 & 0 & 0 & 1 & 0 & 0 & 1 & 0 & 0 & 0 & 0 & 1 & 1 & 1 & 1 & 1 & 1 & 1 & 0 & 0 & 1 & 0 & 0 & 0 & 1 & 1 & 0 & 0 & 0 & 0 & 1 & 0 \\
104P & 0 & 9 & 9 & 4 & 0 & 2 & 0 & 8 & 8 & 2 & 4 & 3 & 3 & 4 & 2 & 0 & 3 & 2 & 6 & 5 & 3 & 3 & 0 & 6 & 1 & 5 & 0 & 6 & 7 & 8 & 2 & 6 & 4 & 9 & 1 & 3 & 0 & 2 & 4 & 5 \\
\hline
105N & 0 & 1 & 1 & 0 & 1 & 1 & 0 & 0 & 1 & 0 & 0 & 0 & 0 & 0 & 0 & 0 & 0 & 1 & 0 & 0 & 0 & 1 & 1 & 1 & 1 & 1 & 1 & 0 & 1 & 1 & 0 & 1 & 1 & 0 & 0 & 0 & 1 & 1 & 0 & 0 \\
105P & 3 & 1 & 8 & 9 & 5 & 4 & 6 & 9 & 9 & 8 & 5 & 7 & 6 & 9 & 8 & 2 & 3 & 9 & 4 & 1 & 5 & 3 & 9 & 6 & 0 & 0 & 5 & 8 & 0 & 8 & 9 & 7 & 5 & 7 & 9 & 9 & 3 & 0 & 8 & 3 \\
\hline
106N & 0 & 0 & 1 & 0 & 0 & 0 & 1 & 0 & 1 & 1 & 0 & 0 & 1 & 1 & 1 & 0 & 1 & 1 & 1 & 0 & 0 & 0 & 1 & 1 & 0 & 1 & 1 & 1 & 1 & 0 & 1 & 0 & 1 & 0 & 0 & 0 & 1 & 1 & 1 & 1 \\
106P & 4 & 9 & 1 & 5 & 1 & 7 & 5 & 2 & 1 & 0 & 3 & 5 & 6 & 7 & 9 & 6 & 2 & 5 & 8 & 6 & 3 & 3 & 4 & 7 & 4 & 4 & 1 & 8 & 6 & 6 & 8 & 9 & 8 & 1 & 6 & 7 & 9 & 6 & 0 & 6 \\
\hline
107N & 0 & 1 & 1 & 0 & 0 & 1 & 0 & 0 & 0 & 1 & 1 & 0 & 0 & 1 & 1 & 1 & 0 & 0 & 1 & 1 & 1 & 1 & 0 & 0 & 1 & 0 & 0 & 1 & 1 & 0 & 1 & 1 & 1 & 1 & 0 & 1 & 0 & 0 & 0 & 1 \\
107P & 7 & 8 & 8 & 5 & 8 & 4 & 4 & 3 & 2 & 0 & 2 & 6 & 6 & 1 & 6 & 4 & 7 & 5 & 5 & 3 & 0 & 2 & 4 & 2 & 8 & 5 & 7 & 8 & 7 & 1 & 9 & 2 & 0 & 7 & 1 & 0 & 5 & 5 & 4 & 7 \\
\hline
108N & 0 & 1 & 1 & 1 & 1 & 1 & 0 & 1 & 0 & 0 & 1 & 1 & 0 & 0 & 0 & 0 & 1 & 1 & 0 & 1 & 1 & 0 & 1 & 0 & 1 & 1 & 0 & 1 & 1 & 0 & 1 & 1 & 1 & 0 & 0 & 1 & 0 & 0 & 0 & 1 \\
108P & 1 & 5 & 1 & 6 & 3 & 2 & 0 & 2 & 3 & 3 & 5 & 5 & 1 & 5 & 1 & 3 & 7 & 4 & 9 & 1 & 2 & 6 & 0 & 6 & 6 & 1 & 7 & 2 & 8 & 6 & 4 & 6 & 0 & 8 & 0 & 5 & 9 & 0 & 7 & 5 \\
\hline
109N & 0 & 1 & 1 & 0 & 0 & 0 & 0 & 1 & 1 & 0 & 1 & 0 & 1 & 1 & 1 & 1 & 0 & 1 & 1 & 0 & 0 & 0 & 0 & 0 & 1 & 0 & 1 & 1 & 1 & 0 & 0 & 0 & 0 & 1 & 1 & 1 & 0 & 0 & 0 & 1 \\
109P & 3 & 9 & 6 & 4 & 1 & 7 & 6 & 5 & 2 & 4 & 4 & 1 & 2 & 5 & 3 & 2 & 1 & 6 & 0 & 9 & 3 & 0 & 6 & 8 & 7 & 4 & 5 & 7 & 4 & 9 & 1 & 2 & 8 & 2 & 7 & 3 & 6 & 6 & 9 & 5 \\
\hline
110N & 1 & 0 & 1 & 1 & 0 & 1 & 0 & 0 & 1 & 0 & 1 & 0 & 0 & 0 & 1 & 0 & 1 & 1 & 0 & 0 & 0 & 1 & 1 & 0 & 0 & 1 & 0 & 0 & 0 & 1 & 0 & 0 & 1 & 1 & 0 & 1 & 1 & 0 & 1 & 1 \\
110P & 8 & 7 & 3 & 4 & 7 & 3 & 3 & 0 & 8 & 4 & 3 & 2 & 9 & 1 & 5 & 2 & 4 & 5 & 2 & 7 & 5 & 6 & 4 & 3 & 6 & 1 & 9 & 2 & 9 & 8 & 9 & 0 & 3 & 6 & 8 & 2 & 1 & 1 & 0 & 6 \\
\hline
111N & 1 & 1 & 1 & 1 & 1 & 1 & 0 & 0 & 0 & 0 & 0 & 1 & 1 & 1 & 0 & 0 & 1 & 1 & 0 & 1 & 1 & 1 & 0 & 0 & 0 & 0 & 1 & 1 & 0 & 1 & 1 & 1 & 1 & 1 & 1 & 1 & 0 & 0 & 0 & 1 \\
111P & 4 & 5 & 7 & 2 & 9 & 9 & 9 & 4 & 0 & 6 & 5 & 1 & 4 & 1 & 0 & 2 & 3 & 9 & 2 & 9 & 3 & 5 & 8 & 2 & 5 & 9 & 9 & 2 & 4 & 1 & 9 & 7 & 3 & 1 & 5 & 8 & 3 & 0 & 4 & 3 \\
\hline
112N & 1 & 1 & 1 & 1 & 0 & 0 & 1 & 1 & 0 & 1 & 0 & 0 & 0 & 1 & 1 & 1 & 0 & 0 & 0 & 1 & 0 & 0 & 1 & 0 & 0 & 1 & 1 & 0 & 0 & 1 & 1 & 0 & 0 & 0 & 0 & 1 & 1 & 1 & 1 & 1 \\
112P & 9 & 9 & 6 & 8 & 5 & 8 & 8 & 7 & 4 & 3 & 2 & 5 & 5 & 1 & 4 & 5 & 9 & 8 & 0 & 9 & 0 & 3 & 0 & 0 & 2 & 9 & 5 & 1 & 3 & 9 & 6 & 7 & 9 & 4 & 8 & 9 & 4 & 3 & 3 & 1 \\
\hline
113N & 0 & 0 & 1 & 0 & 0 & 1 & 1 & 0 & 0 & 0 & 0 & 1 & 1 & 1 & 0 & 0 & 0 & 0 & 1 & 1 & 1 & 1 & 0 & 0 & 1 & 0 & 1 & 0 & 0 & 0 & 1 & 1 & 0 & 1 & 1 & 1 & 1 & 0 & 1 & 0 \\
113P & 7 & 6 & 1 & 8 & 5 & 2 & 6 & 5 & 0 & 3 & 0 & 9 & 5 & 2 & 3 & 5 & 6 & 5 & 1 & 2 & 1 & 7 & 5 & 9 & 7 & 6 & 4 & 6 & 9 & 4 & 0 & 8 & 9 & 0 & 7 & 8 & 4 & 2 & 1 & 0 \\
\hline
114N & 1 & 1 & 0 & 0 & 0 & 0 & 1 & 0 & 0 & 0 & 0 & 0 & 1 & 0 & 0 & 0 & 0 & 0 & 1 & 0 & 1 & 0 & 1 & 0 & 0 & 0 & 1 & 1 & 1 & 1 & 1 & 1 & 0 & 1 & 0 & 1 & 1 & 0 & 1 & 0 \\
114P & 1 & 1 & 6 & 6 & 4 & 7 & 2 & 7 & 6 & 2 & 0 & 5 & 4 & 5 & 3 & 4 & 1 & 2 & 8 & 0 & 0 & 3 & 1 & 5 & 8 & 3 & 0 & 4 & 9 & 4 & 5 & 2 & 9 & 0 & 3 & 3 & 8 & 6 & 7 & 7 \\
\hline
115N & 0 & 0 & 0 & 1 & 1 & 1 & 0 & 0 & 0 & 0 & 0 & 1 & 1 & 1 & 1 & 0 & 0 & 1 & 0 & 1 & 0 & 0 & 1 & 0 & 0 & 0 & 0 & 0 & 0 & 1 & 0 & 1 & 0 & 1 & 0 & 0 & 1 & 1 & 0 & 0 \\
115P & 3 & 4 & 9 & 4 & 7 & 3 & 6 & 0 & 8 & 8 & 8 & 3 & 1 & 0 & 2 & 8 & 0 & 1 & 9 & 6 & 1 & 0 & 2 & 2 & 1 & 3 & 4 & 5 & 9 & 5 & 4 & 0 & 0 & 3 & 2 & 6 & 3 & 1 & 5 & 3 \\
\hline
116N & 1 & 1 & 1 & 0 & 0 & 1 & 0 & 0 & 0 & 0 & 0 & 1 & 1 & 1 & 0 & 0 & 1 & 1 & 1 & 0 & 0 & 1 & 1 & 0 & 1 & 0 & 0 & 1 & 0 & 0 & 0 & 0 & 0 & 0 & 0 & 0 & 1 & 0 & 0 & 0 \\
116P & 2 & 1 & 5 & 2 & 3 & 1 & 4 & 5 & 6 & 5 & 5 & 8 & 6 & 8 & 0 & 5 & 0 & 8 & 2 & 8 & 2 & 6 & 4 & 2 & 3 & 1 & 4 & 4 & 4 & 9 & 1 & 7 & 8 & 5 & 3 & 9 & 2 & 2 & 6 & 3 \\
\hline
117N & 1 & 0 & 0 & 0 & 1 & 1 & 0 & 0 & 1 & 0 & 1 & 0 & 1 & 1 & 0 & 1 & 1 & 0 & 0 & 1 & 1 & 1 & 1 & 0 & 0 & 1 & 1 & 0 & 1 & 1 & 0 & 1 & 0 & 1 & 0 & 1 & 0 & 1 & 0 & 0 \\
117P & 1 & 0 & 3 & 1 & 9 & 6 & 3 & 2 & 0 & 5 & 3 & 0 & 3 & 3 & 8 & 8 & 1 & 2 & 7 & 9 & 8 & 9 & 3 & 9 & 9 & 7 & 3 & 7 & 2 & 4 & 2 & 8 & 2 & 2 & 1 & 4 & 2 & 1 & 3 & 1 \\
\hline
118N & 1 & 0 & 0 & 0 & 1 & 1 & 0 & 0 & 0 & 0 & 1 & 0 & 1 & 0 & 1 & 0 & 1 & 0 & 1 & 1 & 0 & 0 & 0 & 1 & 0 & 1 & 0 & 0 & 0 & 0 & 0 & 1 & 1 & 1 & 0 & 1 & 1 & 0 & 1 & 1 \\
118P & 6 & 7 & 0 & 3 & 3 & 9 & 1 & 8 & 1 & 3 & 5 & 7 & 8 & 5 & 1 & 3 & 9 & 9 & 3 & 2 & 9 & 8 & 8 & 9 & 8 & 3 & 2 & 7 & 5 & 9 & 5 & 3 & 5 & 5 & 3 & 5 & 8 & 3 & 3 & 4 \\
\hline
119N & 1 & 1 & 1 & 0 & 0 & 1 & 0 & 1 & 1 & 0 & 1 & 0 & 0 & 0 & 0 & 1 & 1 & 0 & 0 & 0 & 1 & 0 & 0 & 0 & 1 & 1 & 1 & 1 & 1 & 0 & 1 & 1 & 1 & 1 & 0 & 1 & 1 & 0 & 1 & 0 \\
119P & 0 & 2 & 1 & 8 & 7 & 3 & 0 & 0 & 7 & 2 & 8 & 8 & 4 & 9 & 6 & 2 & 2 & 5 & 2 & 0 & 9 & 6 & 1 & 7 & 3 & 0 & 3 & 5 & 9 & 8 & 5 & 6 & 0 & 6 & 7 & 9 & 3 & 0 & 9 & 5 \\
\hline
120N & 0 & 1 & 0 & 1 & 1 & 1 & 0 & 1 & 0 & 1 & 1 & 0 & 1 & 0 & 1 & 1 & 0 & 1 & 0 & 1 & 1 & 1 & 0 & 1 & 0 & 0 & 0 & 1 & 0 & 0 & 0 & 0 & 0 & 0 & 1 & 1 & 0 & 1 & 1 & 0 \\
120P & 6 & 8 & 9 & 6 & 4 & 4 & 5 & 6 & 7 & 1 & 1 & 6 & 2 & 9 & 8 & 4 & 7 & 6 & 1 & 9 & 5 & 6 & 0 & 2 & 6 & 1 & 9 & 9 & 5 & 3 & 5 & 1 & 3 & 6 & 9 & 5 & 7 & 0 & 4 & 5 \\
\hline
121N & 0 & 0 & 1 & 1 & 1 & 0 & 1 & 0 & 0 & 1 & 0 & 1 & 0 & 0 & 1 & 0 & 1 & 1 & 0 & 0 & 0 & 0 & 1 & 1 & 1 & 1 & 0 & 0 & 1 & 1 & 0 & 1 & 0 & 0 & 0 & 0 & 0 & 1 & 1 & 0 \\
121P & 0 & 8 & 0 & 5 & 8 & 3 & 2 & 5 & 5 & 2 & 8 & 9 & 3 & 7 & 4 & 9 & 7 & 5 & 0 & 3 & 0 & 0 & 9 & 5 & 5 & 9 & 2 & 7 & 7 & 1 & 9 & 8 & 1 & 5 & 2 & 1 & 5 & 9 & 5 & 7 \\
\hline
122N & 0 & 0 & 1 & 1 & 0 & 1 & 1 & 1 & 1 & 0 & 0 & 1 & 0 & 0 & 0 & 0 & 1 & 1 & 1 & 0 & 1 & 1 & 1 & 0 & 0 & 1 & 0 & 0 & 0 & 1 & 1 & 0 & 1 & 1 & 0 & 1 & 1 & 1 & 1 & 1 \\
122P & 7 & 6 & 6 & 1 & 5 & 0 & 1 & 8 & 9 & 3 & 9 & 8 & 1 & 8 & 4 & 4 & 1 & 2 & 0 & 7 & 8 & 9 & 1 & 8 & 4 & 4 & 3 & 1 & 8 & 2 & 4 & 1 & 6 & 4 & 1 & 9 & 5 & 1 & 2 & 4 \\
\hline
123N & 1 & 0 & 0 & 0 & 1 & 1 & 0 & 1 & 1 & 1 & 1 & 1 & 0 & 1 & 0 & 1 & 0 & 0 & 0 & 0 & 0 & 0 & 1 & 0 & 0 & 1 & 1 & 0 & 1 & 0 & 0 & 1 & 1 & 0 & 1 & 1 & 1 & 0 & 1 & 1 \\
123P & 9 & 9 & 4 & 0 & 9 & 1 & 5 & 5 & 2 & 6 & 8 & 7 & 2 & 8 & 3 & 6 & 5 & 6 & 6 & 0 & 9 & 5 & 1 & 4 & 8 & 8 & 0 & 7 & 8 & 8 & 8 & 6 & 0 & 6 & 8 & 1 & 2 & 3 & 1 & 3 \\
\hline
124N & 1 & 0 & 1 & 1 & 0 & 0 & 0 & 0 & 0 & 0 & 0 & 0 & 0 & 1 & 1 & 1 & 1 & 0 & 1 & 1 & 1 & 0 & 1 & 0 & 1 & 0 & 1 & 0 & 1 & 0 & 1 & 0 & 0 & 1 & 0 & 0 & 1 & 0 & 1 & 0 \\
124P & 7 & 2 & 3 & 3 & 8 & 0 & 4 & 7 & 6 & 9 & 7 & 7 & 6 & 0 & 2 & 1 & 4 & 3 & 2 & 6 & 3 & 8 & 6 & 1 & 0 & 1 & 8 & 3 & 0 & 8 & 0 & 4 & 3 & 0 & 2 & 9 & 7 & 8 & 1 & 4 \\
\hline
125N & 0 & 0 & 1 & 1 & 1 & 1 & 0 & 0 & 1 & 1 & 1 & 1 & 0 & 1 & 1 & 1 & 0 & 1 & 0 & 0 & 1 & 0 & 0 & 1 & 1 & 0 & 0 & 0 & 1 & 1 & 0 & 1 & 1 & 1 & 0 & 0 & 0 & 1 & 0 & 0 \\
125P & 5 & 0 & 1 & 9 & 0 & 0 & 6 & 5 & 9 & 8 & 0 & 9 & 4 & 6 & 5 & 0 & 6 & 6 & 0 & 9 & 1 & 1 & 3 & 2 & 0 & 6 & 4 & 0 & 9 & 4 & 6 & 8 & 6 & 3 & 2 & 8 & 7 & 8 & 6 & 9 \\
\hline
126N & 1 & 0 & 1 & 0 & 1 & 1 & 0 & 1 & 1 & 1 & 1 & 0 & 0 & 1 & 0 & 0 & 0 & 0 & 1 & 0 & 1 & 0 & 1 & 0 & 1 & 0 & 0 & 1 & 1 & 1 & 1 & 1 & 0 & 1 & 1 & 1 & 0 & 1 & 0 & 1 \\
126P & 6 & 2 & 2 & 0 & 8 & 0 & 6 & 2 & 5 & 5 & 0 & 2 & 3 & 3 & 2 & 6 & 8 & 3 & 3 & 8 & 1 & 0 & 5 & 7 & 3 & 6 & 2 & 4 & 4 & 7 & 4 & 4 & 4 & 8 & 8 & 6 & 0 & 8 & 1 & 5 \\
\hline
127N & 0 & 0 & 0 & 1 & 0 & 0 & 0 & 0 & 0 & 1 & 1 & 1 & 1 & 1 & 0 & 1 & 0 & 0 & 1 & 0 & 0 & 1 & 1 & 0 & 1 & 0 & 0 & 0 & 0 & 0 & 1 & 0 & 0 & 1 & 0 & 1 & 0 & 0 & 1 & 0 \\
127P & 5 & 1 & 9 & 9 & 7 & 8 & 9 & 6 & 5 & 6 & 0 & 8 & 6 & 3 & 1 & 7 & 5 & 9 & 4 & 5 & 1 & 1 & 1 & 8 & 1 & 2 & 1 & 7 & 6 & 1 & 8 & 0 & 0 & 2 & 1 & 6 & 2 & 9 & 2 & 8 \\
\hline
128N & 0 & 0 & 0 & 0 & 0 & 1 & 0 & 1 & 0 & 0 & 0 & 0 & 0 & 0 & 0 & 0 & 0 & 1 & 1 & 1 & 0 & 1 & 1 & 1 & 0 & 0 & 0 & 1 & 1 & 1 & 0 & 1 & 1 & 1 & 1 & 0 & 0 & 0 & 0 & 1 \\
128P & 7 & 8 & 9 & 7 & 4 & 7 & 1 & 5 & 8 & 1 & 5 & 6 & 1 & 4 & 6 & 3 & 3 & 5 & 5 & 4 & 6 & 3 & 3 & 9 & 8 & 1 & 7 & 5 & 3 & 3 & 6 & 9 & 7 & 7 & 3 & 2 & 7 & 8 & 6 & 7 \\
\hline
129N & 0 & 0 & 1 & 1 & 1 & 1 & 1 & 1 & 1 & 0 & 0 & 0 & 1 & 0 & 0 & 1 & 1 & 0 & 1 & 1 & 1 & 0 & 1 & 0 & 0 & 0 & 0 & 1 & 0 & 0 & 1 & 0 & 0 & 0 & 0 & 0 & 0 & 1 & 0 & 1 \\
129P & 5 & 4 & 5 & 7 & 6 & 9 & 3 & 8 & 5 & 6 & 1 & 9 & 5 & 2 & 4 & 8 & 8 & 5 & 3 & 3 & 5 & 9 & 5 & 8 & 2 & 8 & 4 & 8 & 4 & 8 & 7 & 5 & 2 & 8 & 4 & 5 & 8 & 6 & 0 & 7 \\
\hline
130N & 0 & 0 & 1 & 0 & 1 & 1 & 1 & 0 & 1 & 1 & 1 & 1 & 0 & 1 & 0 & 1 & 1 & 0 & 0 & 1 & 1 & 0 & 0 & 1 & 1 & 1 & 0 & 0 & 0 & 0 & 0 & 1 & 1 & 1 & 1 & 1 & 1 & 1 & 0 & 0 \\
130P & 3 & 2 & 3 & 0 & 6 & 3 & 0 & 2 & 4 & 8 & 3 & 8 & 9 & 0 & 3 & 3 & 5 & 0 & 6 & 5 & 2 & 9 & 0 & 7 & 7 & 6 & 1 & 6 & 3 & 7 & 2 & 0 & 0 & 5 & 6 & 3 & 6 & 6 & 5 & 2 \\
\hline
131N & 0 & 1 & 0 & 1 & 1 & 1 & 1 & 1 & 1 & 1 & 1 & 0 & 1 & 1 & 0 & 1 & 1 & 0 & 1 & 0 & 0 & 1 & 0 & 1 & 0 & 1 & 1 & 1 & 0 & 0 & 0 & 1 & 0 & 1 & 1 & 0 & 1 & 0 & 1 & 0 \\
131P & 6 & 7 & 0 & 6 & 4 & 7 & 7 & 8 & 3 & 2 & 7 & 7 & 3 & 9 & 5 & 1 & 4 & 7 & 9 & 5 & 2 & 0 & 2 & 4 & 9 & 6 & 6 & 5 & 4 & 3 & 9 & 7 & 9 & 3 & 4 & 7 & 0 & 4 & 0 & 9 \\
\hline
132N & 0 & 1 & 1 & 0 & 0 & 0 & 0 & 1 & 1 & 0 & 0 & 0 & 1 & 0 & 0 & 0 & 0 & 1 & 1 & 1 & 1 & 1 & 1 & 0 & 0 & 0 & 0 & 0 & 0 & 1 & 0 & 0 & 1 & 0 & 0 & 1 & 0 & 1 & 1 & 0 \\
132P & 5 & 9 & 1 & 7 & 8 & 7 & 3 & 2 & 7 & 1 & 1 & 2 & 8 & 6 & 4 & 1 & 6 & 0 & 3 & 8 & 2 & 3 & 3 & 6 & 4 & 8 & 1 & 0 & 1 & 5 & 8 & 9 & 9 & 2 & 7 & 0 & 3 & 9 & 4 & 2 \\
\hline
133N & 0 & 1 & 0 & 0 & 1 & 1 & 0 & 1 & 0 & 1 & 1 & 1 & 1 & 1 & 1 & 1 & 1 & 0 & 1 & 0 & 1 & 0 & 1 & 0 & 0 & 1 & 0 & 1 & 1 & 1 & 0 & 0 & 0 & 0 & 1 & 1 & 1 & 0 & 1 & 0 \\
133P & 5 & 3 & 7 & 6 & 9 & 4 & 8 & 8 & 1 & 8 & 6 & 4 & 2 & 6 & 9 & 0 & 2 & 1 & 8 & 3 & 8 & 9 & 1 & 8 & 5 & 2 & 0 & 3 & 1 & 8 & 5 & 5 & 8 & 0 & 0 & 9 & 5 & 7 & 8 & 3 \\
\hline
134N & 0 & 1 & 0 & 0 & 0 & 1 & 1 & 0 & 0 & 1 & 0 & 0 & 0 & 0 & 0 & 0 & 1 & 1 & 0 & 0 & 0 & 1 & 0 & 0 & 0 & 1 & 0 & 0 & 0 & 1 & 1 & 0 & 1 & 1 & 0 & 0 & 1 & 0 & 1 & 1 \\
134P & 4 & 9 & 3 & 0 & 6 & 7 & 0 & 5 & 9 & 6 & 0 & 8 & 6 & 8 & 6 & 7 & 2 & 7 & 7 & 3 & 9 & 5 & 6 & 8 & 8 & 3 & 3 & 2 & 2 & 8 & 4 & 2 & 1 & 4 & 7 & 4 & 7 & 9 & 4 & 9 \\
\hline
135N & 1 & 0 & 0 & 0 & 1 & 1 & 0 & 1 & 1 & 0 & 0 & 1 & 1 & 0 & 0 & 1 & 0 & 0 & 0 & 0 & 0 & 1 & 0 & 0 & 1 & 1 & 1 & 1 & 0 & 0 & 0 & 1 & 0 & 0 & 1 & 0 & 1 & 1 & 1 & 0 \\
135P & 9 & 9 & 4 & 0 & 4 & 4 & 3 & 5 & 7 & 4 & 2 & 6 & 7 & 2 & 1 & 8 & 1 & 7 & 4 & 3 & 8 & 8 & 0 & 0 & 0 & 4 & 1 & 0 & 7 & 8 & 1 & 9 & 8 & 7 & 9 & 8 & 9 & 9 & 0 & 5 \\
\hline
136N & 0 & 0 & 0 & 0 & 1 & 1 & 0 & 1 & 0 & 0 & 0 & 0 & 0 & 1 & 1 & 0 & 1 & 0 & 1 & 0 & 0 & 0 & 1 & 1 & 1 & 0 & 0 & 1 & 1 & 1 & 0 & 1 & 0 & 1 & 1 & 1 & 0 & 0 & 1 & 1 \\
136P & 7 & 0 & 3 & 6 & 7 & 9 & 0 & 3 & 7 & 5 & 8 & 4 & 4 & 9 & 4 & 5 & 4 & 7 & 6 & 4 & 4 & 5 & 7 & 1 & 5 & 1 & 6 & 7 & 5 & 4 & 2 & 9 & 7 & 0 & 1 & 7 & 7 & 5 & 2 & 5 \\
\hline
137N & 0 & 1 & 1 & 1 & 1 & 0 & 0 & 0 & 1 & 1 & 0 & 0 & 0 & 0 & 1 & 0 & 0 & 1 & 0 & 0 & 1 & 0 & 0 & 0 & 0 & 0 & 0 & 0 & 1 & 1 & 0 & 0 & 0 & 1 & 1 & 0 & 0 & 0 & 0 & 1 \\
137P & 7 & 2 & 8 & 0 & 4 & 1 & 2 & 5 & 3 & 0 & 4 & 7 & 9 & 6 & 8 & 0 & 1 & 0 & 4 & 6 & 8 & 2 & 5 & 2 & 3 & 3 & 2 & 2 & 1 & 2 & 7 & 2 & 0 & 8 & 8 & 8 & 5 & 4 & 9 & 5 \\
\hline
138N & 1 & 0 & 0 & 1 & 1 & 1 & 0 & 0 & 0 & 1 & 1 & 1 & 0 & 1 & 1 & 0 & 1 & 0 & 0 & 1 & 1 & 1 & 1 & 0 & 1 & 1 & 1 & 0 & 0 & 1 & 0 & 0 & 0 & 0 & 1 & 1 & 0 & 0 & 1 & 1 \\
138P & 4 & 7 & 4 & 0 & 5 & 6 & 3 & 6 & 9 & 1 & 7 & 4 & 5 & 8 & 7 & 2 & 6 & 1 & 1 & 6 & 7 & 7 & 0 & 4 & 7 & 8 & 5 & 3 & 0 & 4 & 6 & 2 & 5 & 9 & 4 & 6 & 6 & 2 & 1 & 3 \\
\hline
139N & 1 & 0 & 0 & 1 & 0 & 1 & 1 & 1 & 0 & 1 & 0 & 0 & 0 & 0 & 1 & 1 & 1 & 1 & 0 & 1 & 0 & 0 & 0 & 1 & 1 & 1 & 1 & 0 & 0 & 0 & 1 & 1 & 0 & 0 & 0 & 1 & 0 & 0 & 0 & 0 \\
139P & 1 & 3 & 4 & 2 & 6 & 4 & 6 & 5 & 4 & 0 & 0 & 5 & 5 & 3 & 7 & 4 & 7 & 5 & 7 & 7 & 0 & 6 & 7 & 2 & 5 & 6 & 0 & 3 & 2 & 2 & 2 & 6 & 3 & 8 & 2 & 0 & 1 & 9 & 1 & 0 \\
\hline
140N & 0 & 1 & 1 & 1 & 1 & 1 & 0 & 0 & 0 & 0 & 1 & 1 & 0 & 1 & 0 & 1 & 0 & 1 & 1 & 1 & 0 & 1 & 1 & 1 & 1 & 0 & 0 & 0 & 0 & 1 & 0 & 0 & 0 & 1 & 0 & 1 & 0 & 1 & 1 & 1 \\
140P & 1 & 9 & 1 & 3 & 6 & 0 & 3 & 4 & 9 & 1 & 3 & 6 & 5 & 5 & 9 & 6 & 8 & 5 & 5 & 6 & 8 & 7 & 3 & 5 & 9 & 7 & 1 & 1 & 3 & 7 & 6 & 1 & 9 & 3 & 8 & 9 & 7 & 9 & 6 & 9 \\
\hline
141N & 1 & 0 & 1 & 0 & 0 & 0 & 0 & 0 & 1 & 0 & 1 & 1 & 0 & 1 & 1 & 1 & 1 & 1 & 1 & 1 & 0 & 0 & 0 & 1 & 0 & 1 & 0 & 1 & 0 & 0 & 0 & 1 & 1 & 0 & 1 & 0 & 1 & 0 & 1 & 1 \\
141P & 1 & 3 & 6 & 4 & 5 & 4 & 0 & 9 & 7 & 7 & 1 & 0 & 8 & 6 & 6 & 4 & 8 & 7 & 1 & 8 & 0 & 6 & 6 & 3 & 0 & 0 & 3 & 6 & 2 & 6 & 6 & 0 & 6 & 3 & 2 & 5 & 0 & 1 & 2 & 1 \\
\hline
142N & 0 & 1 & 1 & 1 & 1 & 0 & 0 & 0 & 1 & 0 & 0 & 1 & 0 & 0 & 0 & 0 & 1 & 0 & 1 & 0 & 1 & 0 & 1 & 0 & 0 & 0 & 0 & 1 & 1 & 1 & 0 & 0 & 1 & 0 & 0 & 1 & 1 & 1 & 0 & 0 \\
142P & 6 & 7 & 1 & 4 & 2 & 2 & 4 & 9 & 4 & 1 & 5 & 2 & 2 & 1 & 3 & 6 & 4 & 8 & 3 & 0 & 3 & 1 & 8 & 9 & 3 & 3 & 9 & 8 & 3 & 7 & 6 & 0 & 3 & 2 & 8 & 3 & 2 & 4 & 2 & 8 \\
\hline
143N & 1 & 0 & 0 & 1 & 0 & 1 & 0 & 0 & 0 & 1 & 1 & 0 & 0 & 1 & 0 & 0 & 1 & 0 & 1 & 1 & 0 & 1 & 1 & 0 & 0 & 1 & 1 & 1 & 0 & 0 & 1 & 1 & 0 & 0 & 0 & 1 & 1 & 1 & 1 & 1 \\
143P & 7 & 9 & 4 & 8 & 3 & 8 & 3 & 2 & 5 & 6 & 2 & 5 & 5 & 9 & 8 & 6 & 0 & 4 & 9 & 2 & 2 & 2 & 1 & 7 & 2 & 3 & 5 & 6 & 9 & 3 & 5 & 8 & 0 & 7 & 5 & 8 & 5 & 4 & 0 & 6 \\
\hline
144N & 1 & 1 & 1 & 1 & 0 & 0 & 1 & 0 & 1 & 1 & 1 & 0 & 1 & 0 & 0 & 1 & 1 & 1 & 0 & 0 & 1 & 1 & 0 & 1 & 1 & 1 & 1 & 1 & 0 & 0 & 1 & 0 & 1 & 0 & 1 & 1 & 1 & 1 & 1 & 1 \\
144P & 7 & 0 & 5 & 1 & 0 & 4 & 8 & 5 & 0 & 5 & 2 & 3 & 0 & 3 & 2 & 6 & 4 & 1 & 7 & 2 & 6 & 6 & 6 & 8 & 0 & 6 & 4 & 7 & 8 & 1 & 0 & 1 & 7 & 3 & 0 & 8 & 7 & 4 & 3 & 6 \\
\hline
145N & 0 & 1 & 1 & 1 & 0 & 1 & 1 & 0 & 0 & 1 & 1 & 1 & 1 & 1 & 1 & 0 & 1 & 0 & 0 & 1 & 0 & 0 & 0 & 0 & 1 & 1 & 0 & 1 & 0 & 0 & 0 & 0 & 0 & 0 & 0 & 0 & 0 & 0 & 1 & 0 \\
145P & 9 & 4 & 0 & 2 & 4 & 5 & 9 & 4 & 4 & 2 & 0 & 0 & 0 & 0 & 0 & 0 & 3 & 5 & 5 & 5 & 1 & 6 & 9 & 2 & 4 & 5 & 5 & 2 & 9 & 4 & 3 & 2 & 6 & 7 & 5 & 5 & 2 & 5 & 5 & 9 \\
\hline
146N & 0 & 1 & 1 & 1 & 0 & 1 & 1 & 1 & 1 & 0 & 0 & 1 & 1 & 1 & 0 & 1 & 1 & 1 & 1 & 0 & 0 & 1 & 0 & 0 & 0 & 1 & 1 & 1 & 1 & 0 & 0 & 1 & 1 & 0 & 1 & 0 & 0 & 1 & 1 & 0 \\
146P & 4 & 9 & 8 & 1 & 9 & 3 & 6 & 8 & 9 & 4 & 2 & 4 & 0 & 2 & 6 & 9 & 5 & 4 & 2 & 4 & 2 & 6 & 9 & 9 & 3 & 6 & 8 & 2 & 7 & 2 & 4 & 5 & 5 & 6 & 7 & 8 & 2 & 5 & 7 & 5 \\
\hline
147N & 1 & 1 & 1 & 0 & 0 & 0 & 1 & 1 & 0 & 0 & 1 & 1 & 0 & 0 & 1 & 1 & 1 & 0 & 1 & 0 & 0 & 0 & 0 & 1 & 0 & 1 & 1 & 0 & 0 & 0 & 0 & 1 & 0 & 0 & 0 & 0 & 1 & 1 & 1 & 0 \\
147P & 3 & 1 & 8 & 7 & 8 & 5 & 3 & 2 & 4 & 8 & 0 & 8 & 7 & 8 & 6 & 6 & 0 & 2 & 2 & 5 & 2 & 8 & 7 & 5 & 0 & 1 & 6 & 9 & 0 & 5 & 4 & 7 & 2 & 2 & 0 & 4 & 8 & 8 & 6 & 8 \\
\hline
148N & 1 & 1 & 0 & 1 & 0 & 0 & 1 & 0 & 1 & 1 & 1 & 0 & 1 & 0 & 0 & 1 & 1 & 1 & 1 & 1 & 1 & 0 & 1 & 0 & 1 & 0 & 0 & 1 & 1 & 1 & 0 & 0 & 0 & 0 & 0 & 1 & 1 & 0 & 0 & 1 \\
148P & 1 & 8 & 7 & 7 & 7 & 3 & 3 & 9 & 1 & 5 & 0 & 2 & 9 & 7 & 4 & 7 & 0 & 2 & 4 & 5 & 7 & 4 & 9 & 8 & 6 & 5 & 6 & 3 & 2 & 8 & 3 & 7 & 1 & 3 & 1 & 3 & 6 & 8 & 2 & 8 \\
\hline
149N & 1 & 0 & 0 & 1 & 1 & 1 & 1 & 1 & 0 & 0 & 0 & 1 & 0 & 1 & 1 & 1 & 0 & 0 & 1 & 0 & 0 & 0 & 0 & 0 & 1 & 1 & 1 & 0 & 1 & 1 & 0 & 0 & 1 & 1 & 1 & 0 & 1 & 1 & 1 & 1 \\
149P & 5 & 7 & 3 & 0 & 8 & 6 & 7 & 0 & 6 & 3 & 8 & 9 & 9 & 5 & 4 & 9 & 3 & 8 & 1 & 7 & 0 & 8 & 3 & 8 & 7 & 6 & 6 & 1 & 4 & 7 & 2 & 6 & 7 & 1 & 2 & 8 & 0 & 5 & 7 & 3 \\
\hline
150N & 0 & 1 & 0 & 1 & 0 & 1 & 1 & 1 & 1 & 0 & 1 & 0 & 0 & 0 & 1 & 1 & 1 & 1 & 1 & 1 & 0 & 1 & 1 & 1 & 0 & 1 & 0 & 0 & 0 & 0 & 1 & 0 & 0 & 0 & 0 & 0 & 0 & 0 & 0 & 1 \\
150P & 1 & 7 & 0 & 5 & 2 & 2 & 2 & 6 & 2 & 6 & 2 & 3 & 9 & 5 & 6 & 6 & 0 & 0 & 5 & 0 & 6 & 9 & 2 & 8 & 9 & 5 & 9 & 5 & 9 & 7 & 8 & 7 & 3 & 1 & 3 & 7 & 8 & 9 & 6 & 0 \\
\hline
151N & 0 & 0 & 1 & 1 & 0 & 1 & 0 & 0 & 1 & 0 & 0 & 1 & 1 & 0 & 1 & 1 & 0 & 1 & 0 & 1 & 1 & 0 & 1 & 0 & 0 & 0 & 0 & 0 & 1 & 0 & 1 & 0 & 0 & 0 & 1 & 0 & 1 & 0 & 0 & 0 \\
151P & 7 & 3 & 5 & 7 & 0 & 9 & 8 & 9 & 2 & 0 & 0 & 3 & 5 & 0 & 1 & 4 & 2 & 4 & 6 & 5 & 5 & 4 & 7 & 1 & 2 & 1 & 4 & 0 & 6 & 9 & 9 & 0 & 2 & 6 & 3 & 8 & 0 & 8 & 2 & 3 \\
\hline
152N & 0 & 0 & 1 & 1 & 1 & 0 & 1 & 1 & 0 & 0 & 0 & 0 & 0 & 0 & 1 & 1 & 1 & 0 & 0 & 0 & 1 & 0 & 1 & 1 & 1 & 1 & 1 & 0 & 1 & 0 & 0 & 0 & 0 & 0 & 0 & 0 & 1 & 0 & 1 & 0 \\
152P & 6 & 0 & 3 & 3 & 9 & 4 & 7 & 6 & 6 & 6 & 6 & 2 & 9 & 1 & 8 & 4 & 2 & 1 & 4 & 8 & 4 & 1 & 5 & 5 & 2 & 6 & 2 & 1 & 3 & 5 & 6 & 9 & 9 & 0 & 9 & 1 & 0 & 6 & 6 & 2 \\
\hline
153N & 1 & 0 & 0 & 0 & 0 & 0 & 0 & 0 & 1 & 0 & 0 & 1 & 1 & 0 & 1 & 0 & 0 & 0 & 0 & 0 & 0 & 0 & 1 & 0 & 0 & 1 & 0 & 1 & 0 & 0 & 1 & 0 & 1 & 0 & 1 & 1 & 0 & 1 & 0 & 0 \\
153P & 9 & 7 & 9 & 6 & 2 & 7 & 2 & 0 & 8 & 2 & 1 & 7 & 3 & 1 & 1 & 7 & 7 & 6 & 7 & 1 & 5 & 6 & 9 & 7 & 7 & 8 & 3 & 3 & 2 & 5 & 4 & 1 & 5 & 3 & 8 & 1 & 2 & 7 & 0 & 5 \\
\hline
154N & 1 & 1 & 0 & 1 & 0 & 0 & 1 & 0 & 1 & 0 & 0 & 0 & 0 & 0 & 1 & 0 & 1 & 1 & 0 & 1 & 0 & 1 & 1 & 0 & 1 & 1 & 1 & 0 & 0 & 1 & 1 & 1 & 1 & 1 & 1 & 0 & 1 & 1 & 1 & 0 \\
154P & 0 & 1 & 8 & 7 & 1 & 3 & 6 & 9 & 9 & 9 & 7 & 6 & 1 & 7 & 3 & 1 & 0 & 6 & 7 & 1 & 4 & 3 & 0 & 9 & 4 & 1 & 5 & 7 & 5 & 0 & 9 & 8 & 5 & 5 & 2 & 5 & 9 & 3 & 7 & 1 \\
\hline
155N & 0 & 1 & 0 & 0 & 1 & 1 & 0 & 0 & 0 & 1 & 1 & 0 & 1 & 1 & 0 & 1 & 1 & 1 & 0 & 0 & 0 & 0 & 0 & 0 & 1 & 1 & 1 & 0 & 0 & 1 & 0 & 0 & 0 & 1 & 0 & 1 & 0 & 1 & 0 & 1 \\
155P & 7 & 1 & 0 & 3 & 8 & 0 & 9 & 3 & 2 & 3 & 5 & 9 & 5 & 6 & 3 & 3 & 9 & 2 & 6 & 1 & 5 & 2 & 2 & 1 & 0 & 3 & 9 & 9 & 5 & 7 & 3 & 8 & 0 & 8 & 9 & 1 & 7 & 1 & 9 & 9 \\
\hline
156N & 0 & 1 & 1 & 0 & 0 & 0 & 0 & 0 & 0 & 0 & 1 & 1 & 1 & 0 & 0 & 0 & 0 & 1 & 1 & 1 & 1 & 1 & 0 & 0 & 1 & 0 & 0 & 0 & 1 & 0 & 1 & 0 & 1 & 1 & 0 & 0 & 0 & 0 & 1 & 0 \\
156P & 7 & 2 & 3 & 4 & 6 & 5 & 4 & 0 & 2 & 1 & 7 & 2 & 8 & 6 & 8 & 7 & 1 & 0 & 2 & 3 & 8 & 0 & 7 & 6 & 5 & 8 & 3 & 9 & 2 & 2 & 4 & 2 & 7 & 7 & 6 & 2 & 4 & 5 & 2 & 0 \\
\hline
157N & 0 & 0 & 0 & 0 & 0 & 1 & 0 & 0 & 1 & 1 & 0 & 0 & 0 & 0 & 0 & 0 & 1 & 1 & 1 & 1 & 0 & 0 & 0 & 1 & 0 & 0 & 0 & 1 & 1 & 1 & 1 & 1 & 0 & 0 & 1 & 1 & 1 & 1 & 1 & 0 \\
157P & 4 & 1 & 4 & 4 & 9 & 6 & 6 & 2 & 9 & 0 & 1 & 6 & 1 & 5 & 3 & 0 & 3 & 5 & 8 & 0 & 1 & 1 & 7 & 2 & 5 & 6 & 9 & 6 & 5 & 5 & 5 & 3 & 9 & 1 & 1 & 3 & 5 & 9 & 3 & 8 \\
\hline
158N & 1 & 0 & 0 & 1 & 1 & 1 & 1 & 1 & 0 & 1 & 0 & 1 & 0 & 1 & 1 & 0 & 1 & 1 & 0 & 0 & 1 & 1 & 1 & 0 & 1 & 1 & 1 & 1 & 1 & 0 & 1 & 1 & 1 & 1 & 0 & 0 & 1 & 0 & 0 & 1 \\
158P & 9 & 4 & 5 & 0 & 9 & 9 & 0 & 9 & 2 & 9 & 4 & 2 & 6 & 5 & 5 & 9 & 9 & 0 & 4 & 8 & 7 & 3 & 5 & 1 & 4 & 9 & 8 & 3 & 4 & 4 & 3 & 7 & 6 & 0 & 9 & 1 & 8 & 5 & 8 & 6 \\
\hline
159N & 0 & 0 & 0 & 0 & 1 & 1 & 0 & 0 & 0 & 0 & 0 & 0 & 0 & 0 & 0 & 1 & 0 & 1 & 1 & 1 & 1 & 1 & 1 & 0 & 0 & 0 & 0 & 0 & 0 & 0 & 1 & 1 & 0 & 0 & 1 & 1 & 1 & 0 & 1 & 1 \\
159P & 3 & 7 & 0 & 7 & 2 & 5 & 7 & 6 & 8 & 5 & 4 & 6 & 5 & 7 & 6 & 3 & 0 & 5 & 6 & 8 & 2 & 5 & 3 & 2 & 4 & 1 & 1 & 5 & 9 & 3 & 0 & 5 & 8 & 3 & 7 & 2 & 1 & 3 & 5 & 1 \\
\hline
160N & 1 & 0 & 0 & 1 & 0 & 1 & 1 & 0 & 1 & 0 & 0 & 1 & 1 & 1 & 1 & 0 & 1 & 0 & 1 & 1 & 1 & 0 & 0 & 0 & 0 & 1 & 0 & 1 & 1 & 1 & 0 & 0 & 1 & 1 & 1 & 0 & 0 & 1 & 1 & 1 \\
160P & 9 & 1 & 2 & 7 & 5 & 5 & 0 & 7 & 6 & 9 & 1 & 9 & 7 & 3 & 8 & 1 & 5 & 4 & 4 & 6 & 9 & 5 & 4 & 1 & 2 & 8 & 8 & 1 & 2 & 2 & 8 & 3 & 2 & 4 & 7 & 7 & 4 & 4 & 7 & 6 \\
\hline
161N & 1 & 0 & 1 & 0 & 0 & 1 & 1 & 0 & 1 & 0 & 0 & 0 & 1 & 1 & 0 & 1 & 1 & 0 & 0 & 1 & 1 & 0 & 0 & 1 & 1 & 1 & 0 & 1 & 1 & 0 & 0 & 0 & 1 & 1 & 0 & 0 & 0 & 0 & 0 & 0 \\
161P & 1 & 4 & 9 & 7 & 6 & 7 & 6 & 9 & 4 & 4 & 5 & 0 & 6 & 5 & 9 & 5 & 9 & 6 & 5 & 7 & 3 & 3 & 6 & 4 & 2 & 8 & 5 & 8 & 9 & 3 & 3 & 4 & 7 & 7 & 5 & 8 & 3 & 4 & 5 & 1 \\
\hline
162N & 1 & 1 & 0 & 1 & 0 & 0 & 1 & 0 & 1 & 0 & 0 & 1 & 0 & 1 & 1 & 1 & 1 & 0 & 1 & 1 & 1 & 0 & 1 & 1 & 0 & 1 & 0 & 0 & 0 & 0 & 1 & 1 & 1 & 1 & 1 & 1 & 1 & 0 & 0 & 0 \\
162P & 4 & 3 & 8 & 8 & 5 & 2 & 4 & 0 & 3 & 9 & 8 & 4 & 0 & 9 & 6 & 1 & 8 & 2 & 0 & 8 & 0 & 2 & 0 & 2 & 0 & 0 & 1 & 2 & 4 & 8 & 2 & 8 & 7 & 0 & 7 & 2 & 6 & 3 & 6 & 6 \\
\hline
163N & 0 & 1 & 1 & 1 & 1 & 1 & 1 & 0 & 1 & 0 & 1 & 1 & 1 & 1 & 1 & 0 & 1 & 0 & 0 & 1 & 1 & 1 & 0 & 0 & 1 & 0 & 0 & 0 & 1 & 0 & 1 & 0 & 1 & 0 & 0 & 0 & 0 & 1 & 0 & 0 \\
163P & 6 & 3 & 5 & 1 & 5 & 3 & 8 & 9 & 4 & 4 & 7 & 8 & 9 & 0 & 6 & 0 & 4 & 5 & 4 & 9 & 5 & 9 & 7 & 2 & 4 & 6 & 6 & 7 & 5 & 5 & 1 & 6 & 0 & 2 & 8 & 4 & 5 & 6 & 6 & 6 \\
\hline
164N & 0 & 0 & 1 & 0 & 1 & 0 & 0 & 0 & 0 & 0 & 1 & 0 & 0 & 1 & 0 & 0 & 0 & 0 & 0 & 0 & 0 & 1 & 1 & 1 & 0 & 0 & 1 & 0 & 0 & 1 & 0 & 0 & 0 & 1 & 1 & 1 & 1 & 1 & 1 & 0 \\
164P & 1 & 1 & 3 & 9 & 8 & 5 & 8 & 8 & 6 & 6 & 5 & 6 & 6 & 0 & 8 & 0 & 3 & 3 & 0 & 4 & 3 & 5 & 6 & 5 & 1 & 4 & 0 & 1 & 7 & 0 & 1 & 8 & 5 & 8 & 4 & 6 & 2 & 7 & 3 & 2 \\
\hline
165N & 0 & 0 & 1 & 0 & 1 & 0 & 1 & 0 & 0 & 0 & 0 & 0 & 0 & 0 & 0 & 1 & 1 & 0 & 0 & 0 & 1 & 1 & 0 & 1 & 0 & 0 & 1 & 1 & 0 & 0 & 0 & 1 & 0 & 1 & 1 & 1 & 0 & 1 & 0 & 0 \\
165P & 8 & 8 & 5 & 0 & 1 & 3 & 2 & 8 & 0 & 1 & 1 & 0 & 7 & 1 & 1 & 1 & 5 & 2 & 9 & 2 & 2 & 3 & 3 & 7 & 9 & 4 & 6 & 9 & 1 & 6 & 4 & 9 & 2 & 1 & 0 & 1 & 9 & 1 & 7 & 2 \\
\hline
166N & 0 & 1 & 0 & 0 & 1 & 1 & 1 & 0 & 1 & 0 & 0 & 1 & 0 & 0 & 1 & 1 & 1 & 1 & 0 & 0 & 1 & 0 & 0 & 1 & 0 & 1 & 1 & 1 & 0 & 1 & 0 & 1 & 0 & 0 & 1 & 0 & 0 & 0 & 1 & 1 \\
166P & 8 & 8 & 1 & 2 & 1 & 3 & 7 & 3 & 3 & 6 & 7 & 9 & 5 & 3 & 4 & 0 & 9 & 4 & 6 & 4 & 9 & 1 & 1 & 8 & 8 & 4 & 6 & 4 & 4 & 1 & 4 & 6 & 5 & 9 & 8 & 9 & 9 & 1 & 5 & 2 \\
\hline
167N & 1 & 0 & 0 & 1 & 1 & 0 & 1 & 0 & 1 & 0 & 1 & 0 & 0 & 0 & 0 & 0 & 0 & 0 & 1 & 1 & 0 & 1 & 0 & 1 & 0 & 1 & 0 & 1 & 0 & 0 & 0 & 1 & 0 & 0 & 1 & 1 & 1 & 1 & 0 & 0 \\
167P & 9 & 2 & 3 & 7 & 3 & 8 & 9 & 1 & 4 & 8 & 5 & 1 & 1 & 1 & 5 & 7 & 8 & 0 & 2 & 4 & 5 & 7 & 2 & 0 & 2 & 9 & 3 & 7 & 4 & 5 & 5 & 1 & 6 & 5 & 9 & 8 & 5 & 3 & 8 & 5 \\
\hline
168N & 0 & 1 & 0 & 0 & 0 & 1 & 1 & 1 & 1 & 0 & 1 & 1 & 0 & 1 & 1 & 0 & 1 & 1 & 0 & 0 & 0 & 0 & 0 & 1 & 0 & 0 & 0 & 0 & 0 & 0 & 0 & 0 & 0 & 1 & 1 & 0 & 1 & 0 & 1 & 1 \\
168P & 1 & 6 & 0 & 4 & 4 & 4 & 3 & 1 & 9 & 2 & 0 & 8 & 4 & 7 & 3 & 4 & 3 & 6 & 4 & 0 & 0 & 5 & 3 & 6 & 3 & 5 & 5 & 2 & 7 & 0 & 4 & 4 & 7 & 0 & 8 & 3 & 8 & 2 & 7 & 8 \\
\hline
169N & 0 & 0 & 0 & 1 & 1 & 1 & 0 & 1 & 0 & 1 & 0 & 0 & 0 & 1 & 0 & 0 & 1 & 0 & 0 & 0 & 0 & 1 & 1 & 0 & 0 & 0 & 1 & 1 & 0 & 0 & 1 & 0 & 1 & 0 & 0 & 1 & 0 & 1 & 1 & 0 \\
169P & 2 & 1 & 6 & 8 & 9 & 0 & 0 & 4 & 9 & 3 & 4 & 2 & 9 & 7 & 9 & 6 & 3 & 1 & 1 & 6 & 3 & 1 & 8 & 6 & 3 & 9 & 1 & 3 & 9 & 4 & 9 & 4 & 5 & 0 & 4 & 9 & 0 & 2 & 4 & 6 \\
\hline
170N & 0 & 1 & 0 & 1 & 0 & 1 & 1 & 0 & 0 & 0 & 1 & 0 & 1 & 0 & 0 & 1 & 1 & 0 & 1 & 1 & 1 & 0 & 1 & 1 & 0 & 0 & 1 & 1 & 1 & 1 & 1 & 0 & 1 & 0 & 0 & 0 & 1 & 1 & 0 & 0 \\
170P & 9 & 0 & 8 & 7 & 1 & 6 & 0 & 4 & 7 & 1 & 7 & 8 & 8 & 4 & 2 & 8 & 4 & 5 & 6 & 8 & 9 & 0 & 6 & 6 & 5 & 8 & 6 & 7 & 2 & 9 & 0 & 1 & 0 & 3 & 7 & 8 & 2 & 2 & 7 & 6 \\
\hline
171N & 1 & 0 & 1 & 0 & 1 & 1 & 1 & 0 & 1 & 1 & 0 & 1 & 1 & 0 & 1 & 1 & 1 & 0 & 1 & 1 & 1 & 1 & 0 & 1 & 1 & 1 & 1 & 1 & 1 & 1 & 1 & 0 & 1 & 0 & 0 & 0 & 1 & 0 & 0 & 0 \\
171P & 1 & 5 & 6 & 7 & 6 & 4 & 4 & 1 & 4 & 4 & 2 & 4 & 4 & 8 & 5 & 9 & 4 & 0 & 5 & 2 & 3 & 7 & 1 & 3 & 3 & 2 & 2 & 1 & 3 & 4 & 5 & 6 & 7 & 7 & 9 & 5 & 3 & 9 & 2 & 8 \\
\hline
172N & 0 & 0 & 1 & 1 & 0 & 0 & 1 & 1 & 1 & 0 & 1 & 1 & 0 & 0 & 1 & 0 & 1 & 1 & 1 & 1 & 1 & 1 & 1 & 1 & 0 & 0 & 0 & 0 & 0 & 0 & 0 & 1 & 0 & 1 & 0 & 0 & 1 & 1 & 1 & 0 \\
172P & 8 & 3 & 3 & 8 & 8 & 4 & 6 & 0 & 3 & 5 & 1 & 1 & 7 & 3 & 5 & 9 & 1 & 7 & 4 & 7 & 5 & 5 & 3 & 9 & 6 & 7 & 3 & 3 & 0 & 3 & 0 & 1 & 9 & 9 & 3 & 8 & 0 & 0 & 0 & 4 \\
\hline
173N & 0 & 1 & 1 & 0 & 1 & 0 & 1 & 1 & 0 & 0 & 0 & 1 & 0 & 0 & 1 & 0 & 0 & 0 & 0 & 0 & 1 & 1 & 0 & 0 & 0 & 1 & 0 & 1 & 0 & 1 & 1 & 1 & 0 & 1 & 0 & 0 & 0 & 0 & 0 & 0 \\
173P & 6 & 2 & 1 & 1 & 9 & 8 & 6 & 1 & 4 & 6 & 5 & 1 & 2 & 1 & 5 & 1 & 4 & 9 & 5 & 3 & 9 & 9 & 0 & 5 & 6 & 1 & 3 & 3 & 2 & 9 & 8 & 9 & 0 & 0 & 5 & 1 & 2 & 6 & 5 & 3 \\
\hline
174N & 0 & 1 & 0 & 0 & 1 & 0 & 1 & 1 & 1 & 0 & 0 & 0 & 0 & 1 & 0 & 0 & 1 & 0 & 1 & 0 & 0 & 1 & 0 & 0 & 0 & 0 & 0 & 0 & 0 & 0 & 1 & 1 & 1 & 1 & 0 & 1 & 1 & 1 & 0 & 1 \\
174P & 4 & 1 & 6 & 8 & 6 & 3 & 8 & 9 & 4 & 6 & 0 & 0 & 6 & 1 & 9 & 6 & 9 & 9 & 4 & 2 & 7 & 8 & 1 & 6 & 0 & 8 & 7 & 8 & 0 & 5 & 7 & 8 & 4 & 4 & 5 & 8 & 7 & 6 & 6 & 9 \\
\hline
175N & 1 & 1 & 1 & 1 & 1 & 1 & 1 & 1 & 0 & 0 & 0 & 1 & 0 & 1 & 0 & 1 & 0 & 0 & 0 & 0 & 0 & 1 & 1 & 0 & 1 & 1 & 0 & 1 & 1 & 1 & 1 & 0 & 0 & 1 & 0 & 1 & 0 & 0 & 0 & 1 \\
175P & 3 & 9 & 4 & 6 & 8 & 0 & 7 & 5 & 1 & 9 & 0 & 8 & 8 & 4 & 9 & 8 & 3 & 2 & 9 & 5 & 1 & 6 & 0 & 3 & 4 & 4 & 8 & 3 & 1 & 5 & 9 & 0 & 2 & 5 & 9 & 2 & 6 & 5 & 1 & 2 \\
\hline
176N & 0 & 0 & 1 & 0 & 1 & 0 & 1 & 1 & 0 & 1 & 0 & 0 & 0 & 1 & 0 & 1 & 1 & 1 & 1 & 1 & 0 & 1 & 1 & 1 & 1 & 0 & 0 & 0 & 1 & 1 & 1 & 0 & 0 & 0 & 0 & 1 & 1 & 0 & 1 & 0 \\
176P & 9 & 3 & 9 & 8 & 9 & 6 & 7 & 2 & 9 & 7 & 6 & 4 & 2 & 2 & 2 & 7 & 3 & 7 & 5 & 7 & 3 & 0 & 4 & 5 & 8 & 6 & 6 & 5 & 7 & 1 & 4 & 8 & 3 & 5 & 2 & 2 & 1 & 4 & 0 & 4 \\
\hline
177N & 0 & 1 & 0 & 1 & 1 & 0 & 1 & 1 & 0 & 1 & 1 & 0 & 0 & 1 & 0 & 1 & 1 & 1 & 1 & 1 & 0 & 1 & 1 & 0 & 0 & 1 & 1 & 0 & 0 & 0 & 0 & 1 & 0 & 0 & 1 & 1 & 1 & 0 & 1 & 0 \\
177P & 2 & 7 & 8 & 3 & 5 & 7 & 4 & 6 & 7 & 4 & 5 & 7 & 6 & 6 & 2 & 4 & 9 & 0 & 0 & 6 & 3 & 7 & 7 & 1 & 2 & 6 & 1 & 4 & 9 & 3 & 3 & 3 & 8 & 2 & 2 & 3 & 1 & 8 & 8 & 8 \\
\hline
178N & 0 & 1 & 1 & 0 & 0 & 0 & 1 & 0 & 1 & 0 & 0 & 0 & 1 & 0 & 1 & 0 & 1 & 1 & 0 & 0 & 0 & 1 & 1 & 0 & 0 & 0 & 1 & 0 & 1 & 0 & 1 & 0 & 0 & 0 & 1 & 0 & 1 & 1 & 0 & 1 \\
178P & 7 & 1 & 3 & 6 & 7 & 9 & 2 & 9 & 8 & 2 & 5 & 4 & 3 & 9 & 2 & 8 & 5 & 6 & 0 & 4 & 4 & 1 & 2 & 3 & 6 & 2 & 6 & 5 & 5 & 3 & 8 & 8 & 6 & 7 & 9 & 1 & 3 & 8 & 6 & 6 \\
\hline
179N & 1 & 1 & 0 & 0 & 1 & 1 & 0 & 0 & 0 & 0 & 1 & 0 & 0 & 1 & 0 & 1 & 1 & 0 & 1 & 1 & 0 & 0 & 0 & 1 & 0 & 1 & 0 & 0 & 0 & 0 & 1 & 0 & 1 & 0 & 1 & 0 & 0 & 0 & 1 & 0 \\
179P & 1 & 3 & 0 & 6 & 7 & 4 & 6 & 0 & 4 & 4 & 1 & 9 & 7 & 9 & 1 & 9 & 3 & 5 & 1 & 0 & 4 & 7 & 0 & 5 & 6 & 0 & 6 & 5 & 0 & 5 & 8 & 0 & 2 & 4 & 7 & 6 & 7 & 7 & 7 & 9 \\
\hline
180N & 1 & 0 & 0 & 0 & 0 & 1 & 1 & 1 & 1 & 0 & 1 & 1 & 1 & 0 & 1 & 0 & 1 & 0 & 0 & 1 & 0 & 1 & 1 & 1 & 0 & 1 & 0 & 1 & 0 & 1 & 0 & 0 & 1 & 0 & 0 & 1 & 0 & 1 & 0 & 0 \\
180P & 3 & 4 & 0 & 8 & 5 & 5 & 0 & 2 & 4 & 7 & 2 & 0 & 5 & 9 & 4 & 9 & 1 & 6 & 0 & 0 & 2 & 3 & 3 & 3 & 5 & 2 & 1 & 9 & 8 & 0 & 2 & 8 & 8 & 2 & 1 & 0 & 0 & 5 & 2 & 2 \\
\hline
181N & 1 & 0 & 1 & 0 & 1 & 0 & 1 & 1 & 1 & 0 & 1 & 1 & 1 & 1 & 0 & 0 & 1 & 0 & 0 & 1 & 0 & 1 & 0 & 0 & 0 & 0 & 1 & 1 & 1 & 1 & 1 & 1 & 0 & 1 & 0 & 0 & 1 & 1 & 1 & 0 \\
181P & 6 & 4 & 4 & 0 & 8 & 4 & 8 & 6 & 1 & 0 & 2 & 2 & 4 & 0 & 9 & 1 & 5 & 1 & 8 & 0 & 5 & 7 & 5 & 0 & 2 & 0 & 1 & 8 & 5 & 3 & 2 & 4 & 4 & 1 & 6 & 7 & 5 & 1 & 1 & 1 \\
\hline
182N & 1 & 1 & 0 & 1 & 1 & 1 & 1 & 1 & 0 & 0 & 1 & 0 & 1 & 1 & 0 & 0 & 0 & 1 & 0 & 1 & 1 & 1 & 0 & 1 & 0 & 0 & 0 & 1 & 0 & 0 & 0 & 1 & 0 & 1 & 0 & 0 & 0 & 0 & 1 & 1 \\
182P & 8 & 3 & 7 & 3 & 0 & 8 & 1 & 7 & 0 & 0 & 1 & 3 & 9 & 5 & 4 & 4 & 5 & 2 & 3 & 2 & 6 & 0 & 8 & 0 & 8 & 8 & 1 & 3 & 8 & 0 & 6 & 6 & 4 & 8 & 0 & 1 & 0 & 5 & 7 & 7 \\
\hline
183N & 1 & 1 & 1 & 0 & 1 & 1 & 0 & 1 & 0 & 1 & 1 & 0 & 0 & 0 & 1 & 1 & 0 & 0 & 0 & 1 & 1 & 1 & 0 & 0 & 1 & 0 & 0 & 1 & 1 & 1 & 0 & 0 & 0 & 1 & 0 & 0 & 0 & 0 & 0 & 1 \\
183P & 3 & 5 & 6 & 2 & 4 & 1 & 4 & 3 & 8 & 9 & 4 & 0 & 0 & 6 & 0 & 3 & 5 & 4 & 2 & 0 & 0 & 3 & 5 & 4 & 8 & 5 & 5 & 6 & 8 & 3 & 8 & 6 & 4 & 6 & 8 & 3 & 6 & 8 & 3 & 9 \\
\hline
184N & 1 & 1 & 1 & 0 & 1 & 0 & 0 & 0 & 1 & 1 & 0 & 1 & 0 & 0 & 1 & 1 & 1 & 0 & 1 & 1 & 0 & 0 & 1 & 1 & 0 & 0 & 0 & 0 & 0 & 1 & 1 & 0 & 0 & 1 & 1 & 1 & 1 & 1 & 0 & 1 \\
184P & 5 & 0 & 5 & 4 & 7 & 6 & 0 & 5 & 3 & 6 & 0 & 0 & 6 & 2 & 4 & 4 & 4 & 8 & 1 & 8 & 7 & 3 & 7 & 6 & 4 & 9 & 8 & 7 & 0 & 2 & 1 & 6 & 4 & 1 & 5 & 0 & 1 & 5 & 7 & 6 \\
\hline
185N & 0 & 1 & 0 & 0 & 1 & 1 & 0 & 0 & 1 & 1 & 1 & 0 & 0 & 1 & 1 & 1 & 1 & 0 & 1 & 1 & 0 & 0 & 0 & 1 & 1 & 0 & 1 & 1 & 1 & 0 & 0 & 0 & 0 & 0 & 0 & 0 & 0 & 1 & 0 & 0 \\
185P & 3 & 2 & 6 & 6 & 6 & 1 & 3 & 2 & 3 & 1 & 4 & 9 & 4 & 0 & 9 & 3 & 4 & 7 & 6 & 4 & 3 & 1 & 0 & 8 & 9 & 5 & 6 & 8 & 4 & 5 & 3 & 8 & 1 & 5 & 2 & 0 & 9 & 6 & 5 & 3 \\
\hline
186N & 0 & 1 & 0 & 0 & 1 & 1 & 1 & 1 & 0 & 0 & 1 & 0 & 0 & 0 & 1 & 1 & 0 & 1 & 0 & 1 & 0 & 0 & 1 & 1 & 0 & 1 & 1 & 0 & 1 & 1 & 0 & 0 & 0 & 0 & 0 & 1 & 0 & 0 & 1 & 0 \\
186P & 9 & 8 & 2 & 0 & 3 & 1 & 3 & 0 & 0 & 7 & 2 & 8 & 8 & 6 & 8 & 7 & 5 & 4 & 0 & 8 & 3 & 9 & 8 & 3 & 5 & 7 & 4 & 4 & 5 & 4 & 2 & 4 & 7 & 0 & 0 & 7 & 0 & 9 & 8 & 2 \\
\hline
187N & 0 & 1 & 0 & 0 & 0 & 1 & 0 & 0 & 0 & 1 & 1 & 1 & 0 & 0 & 1 & 0 & 0 & 0 & 0 & 1 & 0 & 1 & 0 & 0 & 1 & 1 & 0 & 0 & 0 & 0 & 1 & 0 & 0 & 0 & 1 & 0 & 0 & 0 & 1 & 0 \\
187P & 2 & 5 & 2 & 1 & 1 & 6 & 9 & 9 & 1 & 5 & 7 & 2 & 6 & 1 & 8 & 1 & 2 & 7 & 0 & 0 & 7 & 4 & 6 & 6 & 7 & 0 & 5 & 3 & 6 & 6 & 1 & 6 & 7 & 8 & 3 & 1 & 4 & 2 & 9 & 8 \\
\hline
188N & 0 & 0 & 1 & 0 & 0 & 0 & 1 & 0 & 1 & 0 & 0 & 0 & 0 & 0 & 1 & 0 & 0 & 0 & 1 & 1 & 1 & 1 & 0 & 0 & 0 & 1 & 1 & 1 & 0 & 1 & 0 & 0 & 0 & 1 & 0 & 0 & 0 & 0 & 0 & 0 \\
188P & 1 & 0 & 2 & 6 & 8 & 0 & 7 & 6 & 5 & 8 & 1 & 9 & 7 & 7 & 0 & 2 & 8 & 7 & 5 & 1 & 6 & 6 & 9 & 4 & 1 & 8 & 4 & 2 & 4 & 6 & 0 & 4 & 7 & 7 & 3 & 9 & 9 & 2 & 9 & 1 \\
\hline
189N & 1 & 0 & 1 & 1 & 0 & 0 & 1 & 0 & 1 & 1 & 1 & 1 & 0 & 1 & 0 & 1 & 1 & 1 & 1 & 0 & 0 & 1 & 1 & 1 & 1 & 0 & 0 & 1 & 1 & 0 & 1 & 0 & 0 & 1 & 0 & 0 & 1 & 1 & 0 & 1 \\
189P & 5 & 4 & 6 & 7 & 9 & 0 & 6 & 8 & 2 & 6 & 6 & 8 & 7 & 3 & 2 & 7 & 6 & 3 & 4 & 8 & 3 & 0 & 0 & 1 & 9 & 2 & 1 & 9 & 1 & 1 & 6 & 1 & 2 & 8 & 5 & 5 & 7 & 1 & 1 & 4 \\
\hline
190N & 0 & 1 & 0 & 0 & 0 & 1 & 1 & 1 & 0 & 1 & 0 & 1 & 0 & 0 & 1 & 0 & 1 & 0 & 0 & 1 & 1 & 0 & 1 & 1 & 1 & 1 & 1 & 0 & 0 & 1 & 0 & 0 & 1 & 1 & 0 & 0 & 0 & 0 & 1 & 1 \\
190P & 3 & 2 & 3 & 7 & 8 & 5 & 3 & 5 & 8 & 0 & 8 & 2 & 1 & 9 & 8 & 9 & 7 & 9 & 0 & 2 & 3 & 0 & 3 & 0 & 8 & 5 & 2 & 0 & 5 & 5 & 0 & 3 & 5 & 6 & 0 & 1 & 8 & 4 & 8 & 7 \\
\hline
191N & 1 & 1 & 0 & 1 & 1 & 0 & 0 & 1 & 1 & 0 & 0 & 0 & 1 & 1 & 0 & 1 & 0 & 1 & 1 & 0 & 0 & 1 & 1 & 0 & 1 & 1 & 0 & 1 & 0 & 1 & 1 & 0 & 0 & 0 & 0 & 0 & 0 & 1 & 1 & 1 \\
191P & 3 & 0 & 8 & 9 & 5 & 2 & 7 & 0 & 4 & 7 & 3 & 4 & 5 & 3 & 5 & 8 & 4 & 4 & 4 & 6 & 7 & 7 & 6 & 4 & 2 & 1 & 1 & 7 & 1 & 6 & 0 & 1 & 7 & 1 & 7 & 3 & 2 & 7 & 4 & 8 \\
\hline
192N & 0 & 1 & 0 & 0 & 0 & 1 & 1 & 1 & 1 & 1 & 0 & 0 & 1 & 0 & 0 & 1 & 0 & 1 & 1 & 1 & 0 & 1 & 1 & 0 & 1 & 1 & 1 & 0 & 1 & 1 & 1 & 1 & 1 & 1 & 0 & 1 & 0 & 1 & 1 & 0 \\
192P & 1 & 5 & 4 & 8 & 0 & 4 & 4 & 3 & 3 & 0 & 0 & 4 & 4 & 1 & 2 & 3 & 7 & 8 & 3 & 2 & 7 & 6 & 4 & 5 & 1 & 7 & 5 & 8 & 5 & 0 & 3 & 6 & 1 & 3 & 0 & 7 & 4 & 0 & 4 & 3 \\
\hline

\end{longtable}
...continued upto 298 rounds

\section*{Case - More than $1/5$ faulty processes with 40 processes}

When we attempted to run with more than $1/5$ faulty processes, the processes failed to reach consensus and could not make progress beyond certain number of rounds (mostly $1/n$ rounds)

\end{document}